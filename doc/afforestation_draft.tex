% Draft manuscript on forestation in CMIP6 to be submitted to ESD? or similar journal.
% Note: Figures are plotted using .png files for now since SVG rendering is very
% slow for LaTeX. At a later date I will change the figures to .svg files. For example:
%\includesvg[width=\linewidth]{../plots/cVeg_aus_anom.svg}
% Journal candidates:
% - Earth system dynamics
% Preamble.
\documentclass[]{article}
\usepackage[a4paper, total={170mm,257mm}, left=20mm, top=20mm,]{geometry}
\usepackage{svg} % Use SVGs only for final draft.
\usepackage{float} % Allows the [H] option meaning "place figure here and only here"
\usepackage[section]{placeins} % Defines /FloatBarrier so that figures can't move.
\usepackage{subcaption} % Allows subfigures and captions
\usepackage{lineno} % Enable line numbers.
\usepackage{textcomp}
\usepackage[style=authoryear]{biblatex}
\usepackage{amsmath}
\addbibresource{afforestation_references.bib}
%\setlength{\parindent}{1cm}
\linenumbers
\title{Limited mitigation potentials under a high emissions scenario: multi-model and single model ensembles}
\author{Tammas Loughran et al.}

\begin{document}

\maketitle

\begin{center}
    \Large
    \vspace{0.9cm}
    \textbf{Abstract}
\end{center}

\begin{itemize}
    \item Impacts of global scale forestation are still poorly understood as there is wide disgreement between models.
    \item We analyse the results from 8 models of a scenario with high fossil fuel emissions and forest expansion. This scenario demonstrates the ability of only forestation strategies to mitigate climate change under continued increasing CO$_2$ emissions, and includes the potential imacts of increased CO$_2$ concentrationa and a warming climate on vegetation growth.
    \item Model intercomparison shows that there is likely no impact of forestation on global climate under a high global warming scenario.
    \item A set of single-model initial condition ensemble simulations suggests that there is evidence of some increase in local scale warm extremes in response to afforestation, particularly for decreases in cool days.
    \item This is dominated by the increase in albedo as land cover transitions from grasslands and croplands to darker forrests.
    \item Furthermore, we find evidence of shift in the global carbon balance, whereby increased uptake of carbon on the land of ~25 PgC by 2100 and a decrease in the uptake of carbon by the ocean.
    \item This is the first intercomparison of the potentials of forrestation in emissions-driven, fully-coupled earth system models, which includes a fully interactive carbon cycle and climate feedbacks.
\end{itemize}

\raggedright
\parindent=.35in % Setting raggedright removes paragraph indents. This puts them back.

\section{Introduction}

Forests cover approximately 31\% of the global land surface (cite FAO) and are responsible for the removal of 30\% of total anthropogenic emissions from the atmosphere (\cite{friedlingstein_global_2022}).
Forestation is therefore often thought of as a viable strategy to remove CO$_2$ from the atmosphere and mitigate global warming.
In order to effectively limit global warming to below 2°C consistent with the Paris Agreement, there must be not only a reduction of fossil fuel emissions, but also CO$_2$ removal to offset industrial activities that are dificult to decarbonize.
There must also be a shift in the reliance of global diets away from agricultural practices that emit greenhouse gasses into the atmosphere (via e.g. forest clearing or ruminating herbivores).
This shift typically implies widespread abandonment of agricultural lands and an increase in natural vegetation, either as reforestation when damaged forests are restored or as afforestation of lands were not previously occupied by forest.

There have been a number of previous studies on the potential of forestation to remove CO$_2$ from the atmosphere. For example, \cite{sonntag_reforestation_2016} quantified the CO$_2$ removal potential using and Earth System Model, where a high CO$_2$ emissions scenario is simulated (taken from RCP8.5) in combination with the ``forestation" land-use from a low emissions scenario (taken from RCP4.5).
This previous study used the concentration driven CMIP5 version of the MPI-ESM and the CMIP5 representative concentration pathways.
This study aims to quantify the CO$_2$ removal potential of the forestation strategy using CMIP6 models, both at the global and regional scale.
Furthermore, the climate variability uncertainties are quantified from a single model ensemble of the ACCES-ESM1.5 model.

\begin{itemize}
    \item As part of the contributions to the CMIP6 LUMIP experiments, modelling groups were requested to do emissions driven simulations of a similar forestation scenario. This time using the Shared Socioeconomic Pathways' emissions from SSP5-8.5 and the land use of SSP1-2.6.
    \item existing studies on MPI-ESM CMIP5 version using RCP8.5 emissions with RCP4.5 land use. \cite{sonntag_reforestation_2016} \cite{sonntag_quantifying_2018}
    \item ACCESS-ESM1.5 contributed 10 ensemble members to these experiments, and it was the only model to contribute multiple ensemble members to the CMIP6 afforestation experiment.
\end{itemize}

\section{Methods}

\subsection{Experiments}

The CMIP6 experimental design specifies a set of standard simulations called the Diagnostic, Evaluation and Characterization of Klima (DECK) \parencite{eyring_overview_2016}, which are typically used as the foundation to study more specific research questions.
The DECK includes a pre-industrial control simulation (\textit{piControl}) with constant greenhouse gas concentration forcings, and a historical (\textit{historical}) simulation with transient historical greenhouse gas concentration forcings.
Furthermore, the DECK specifies corresponding simulations (\textit{esm-piControl} and \textit{esm-hist}) that are run in full ``Earth system model" mode with a fully interactive dynamical carbon cycle and for the historical period, fossil fuel and industrial greenhouse gas emissions forcings.
In order to capture the carbon cycle and dynamics of the entire Earth system in response to forestation, including biogeochemical feedbacks, we use only the Earth system model simulations from relevant intercomparison projects. These emissions-driven simulations are initialized at 2015 from the end of the \textit{esm-hist} experiment.

The forestation scenario we use is the Land Use Model Intercomparison Project's \textit{esm-ssp585-ssp126Lu}, spanning the years 2015--2100 \parencite{lawrence_land_2016}.
This simulation features the surface fossil fuel related CO$_{2}$ emissions from the SSP5-8.5 scenario, and the land-use change from the SSP1-2.6 scenario \parencite{oneill_scenario_2016}.
The SSP5-8.5 scenario emissions represent a high emissions future pathway that results in a radiative forcing of 8.5 W m$^{-2}$ in 2100.
The SSP1-2.6 scenario land-use change assumes a future of sustainable development \parencite{van_vuuren_energy_2017}.
This means low population growth, low pressures on land-use, environmentally friendly changes in diets and increased agricultural efficiency and yields, resulting in abandonment of agricultural lands.
This allows for the expansion of natural lands and forest cover.

The reference simulation we use to compare to \textit{esm-ssp585-ssp126Lu} is the \textit{esm-ssp585} from the Coupled Climate-Carbon Cycle Model Intercomparison Project (C4MIP) \parencite{jones_c4mip_2016}.
This is the emissions-driven Earth system model simulation corresponding to the SSP5-8.5 high greenhouse gas scenario that assumes that development is driven by fossil fuels \parencite{oneill_scenario_2016}.
In general, the land use change in this scenario has less expansion of forested lands than \textit{esm-ssp585-ssp126Lu}, but is a mixture of moderate forestation and deforestation.
The notable difference in land-use change compared the forestation scenario is the deforestation in the central African region (Figure S\ref{fig:ACCESS_land_cover}).
By taking the difference between the forestation simulation and the \textit{esm-ssp585} simulation (Equation \ref{equ:diff}), we can examine the impact of forestation on the climate and the carbon cycle.

\begin{equation}
    X|_{for} = X|_{esm-ssp585-ssp126Lu} - X|_{esm-ssp585}
    \label{equ:diff}
\end{equation}

\begin{figure}[H]
    \centering
    \begin{subfigure}[b]{0.45\linewidth}
        \includegraphics[width=\linewidth]{../plots/land_areas.png}
    \end{subfigure}
    \begin{subfigure}[b]{0.45\linewidth}

        \includegraphics[width=\linewidth]{../plots/land_areas_diff.png}
    \end{subfigure}
    \caption{Land use areas in ACCESS-ESM1.5. Left: ssp585 (dashed) and forestation scenario (solid). Right: difference between ssp585 and forestation scenario.}
    \label{fig:land_use}
\end{figure}

The land-use change dataset used in all CMIP6 experiments is the Land-use Harmonization dataset version 2 (LUH2; \cite{hurtt_harmonization_2020}).
Each model has a unique representation of woody tree biomes and pre-industrial potential vegetation.
Therefore, each model interprets the distribution of natural lands from LUH2 to plant functional types (PFTs) differently.
Combined with differences in grid resolution, this results in a variety of changes global tree cover for each model (see Figure \ref{fig:CABLE_PFTs}).
As an example, ACCESS-ESM1-5 has by the end of the century a forest expansion of 1.59 million km² and agricultural abandonment of -1.11 million km².
By mid-century crops reach a minimum of 2.74 million km² less than in 2015, before rising again in the latter half of the century (Figure \ref{fig:land_use}).

\begin{figure}[H]
    \centering
    \begin{subfigure}[b]{0.45\linewidth}
        \includegraphics[width=\linewidth]{../plots/treeFrac_anomaly.png}
    \end{subfigure}
    \begin{subfigure}[b]{0.45\linewidth}
        \includegraphics[width=\linewidth]{../plots/grassFrac_anomaly.png}
    \end{subfigure}
    \caption{ACCESS-ESM1.5 land-use area change from 2015 to 2100 in the forestation scenario. Left: forest. Right: Grass. In ACCESS-ESM1.5 LUH2 pasture fractions are allocated to C3 grass.}
    \label{fig:land_use_map}
\end{figure}

\begin{figure}[H]
    \centering
    \begin{subfigure}[b]{0.45\linewidth}
        \includegraphics[width=\linewidth]{../plots/treeFrac_models.png}
    \end{subfigure}
    \begin{subfigure}[b]{0.45\linewidth}
        \includegraphics[width=\linewidth]{../plots/CABLE_forests.png}
    \end{subfigure}
    \caption{Global mean tree cover fraction for each model and CABLE tree PFT areas for the SSP1-2.6 scenario.}
    \label{fig:CABLE_PFTs}
\end{figure}

Forestation is dominated by growth of evergreen broad leaf forests, followed by evergreen needle leaf forests, then deciduous broad leaf forests.
Deciduous needle leaf forests only make a small fraction of forests and do not show any expansion.
There are only a few areas where there is deforestation in the esm-ssp585-ssp126Lu experiment.
These occur in deciduous broad leaf forests in eastern USA, China and western Russia (Figure \ref{fig:land_use_map}).

\subsection{Participating models}

There are seven Earth system models that participated in both LUMIP and C4MIP with simulations available for \textit{esm-ssp585-ssp126Lu} and \textit{esm-ssp585}.
A brief overview of these models are presented in Table \ref{tab:models}.
2 models included wildfire schemes, 3 models included dynamical vegetation, and 6 models included nitrogen nutrient limitaiton.
ACCESS-ESM1-5 is the only model to include phosphorus limitation.
Furthermore, ACCESS-ESM1-5 is the only model to have multiple ensemble members available for both simulations, of which there are 10 for each experiment.

\begin{table}[H]
    \centering
    \begin{tabular}{lllllllll}
        \hline
Model         & Fire & N   & P   & Dyn. Veg. & No. PFT & Nat. PFT & Res. & Ref.                                         \\ \hline
ACCESS-ESM1-5 & No   & Yes & Yes & No    & 10       & 9       & 1.8758×1.258°   & \cite{ziehn_australian_2020}   \\
BCC-CSM2-MR   & No   & Yes & No  & No    & 14       & 11      & 110×110km       & \cite{li_development_2019}     \\
CanESM5       & No   & No  & No  & Yes   & 9        & 7       & ~2.8×2.8°       & \cite{swart_canadian_2019}     \\
GFDL-ESM4     & Yes  & No  & No  & No    & 7        & 5       & 100×100km       & \cite{dunne_gfdl_2020}         \\
MIROC-ES2L    & No   & Yes & No  & No    & 14       & 13      & 2.8×2.8°        & \cite{hajima_development_2020} \\
MPI-ESM1-2-LR & Yes  & Yes & No  & Yes   & 13       & 12      & 1.9°            & \cite{giorgetta_climate_2013}  \\
NorESM2-LM    & Yes  & Yes & No  & No    & 22       & 14      & 2×2°            & \cite{seland_norwegian_2020}   \\
UKESM1-0-LL   & No   & Yes & No  & Yes   & 12       & 8       & 1.25×1.875°     & \cite{sellar_ukesm1_2019}      \\ \hline
    \end{tabular}
    \caption{Models participating in both LUMIP and C4MIP with simulations for both \textit{esm-ssp585-ssp126Lu} and \textit{esm-ssp585}. The columns indicate whether the models land surface component have representations of wildfire, nutrients, dynamical vegetation and the number of plant functional types. Natural plant functional types excludes agricultural crops and pasture.}
    \label{tab:models}
\end{table}

\subsection{Regional analysis}

\begin{figure}[H]
    \centering
    \includegraphics{../plots/regional_analysis_map.png}
    \caption{Box regions for regional analysis. Dots indicate gridpoints that feature large transitions to forest cover and are used in the analysis of the temperature distributions in each simulation.}
    \label{fig:box_regions}
\end{figure}

\subsection{Statistical methods}

We calculate trends in the difference of global mean temperatures and precipitation to examine the change in temperature as forests expand.
For this we use the Theil–Sen slope estimator and test its significance at the 5\% level using the Mann-Kendall trend test.

We also calculate probability histograms to examine the distribution of daily temperatures for the ACCESS-ESM.
The difference in the temperature distribution in the two simulations in response to forestation is tested using 2-sample Kolmogorov-Smirnov test for the equality of distributions at the 5\% level.

\section{Results and Discussion}

\subsection{Model Inter-comparison}

\begin{figure}[H]
    \centering
    \begin{subfigure}[b]{0.45\linewidth}
        \includegraphics[width=\linewidth]{../plots/cLand_model_intercomparison_diff.png}
    \end{subfigure}
    \begin{subfigure}[b]{0.45\linewidth}
        \includegraphics[width=\linewidth]{../plots/cVeg_model_intercomparison_diff.png}
    \end{subfigure}
    \begin{subfigure}[b]{0.45\linewidth}
        \includegraphics[width=\linewidth]{../plots/cSoil+cLitter_model_intercomparison_diff.png}
    \end{subfigure}
    \caption{cLand, cVeg and cLitter+cSoil in esm-ssp585-ssp126Lu scenario for 6 CMIP6 models. ACCESS-ESM is plotted as the ensemble mean and the blue shading indicates the ACCESS-ESM1-5 ensemble range.}
    \label{fig:models_cpools}
\end{figure}

\begin{itemize}
    \item CanESM5 has differing ensemble methods for esm-ssp585 and esm-ssp585-ssp126Lu. One uses r1i1p1f1 and another uses r1i1p2f1. Therefore, there is some additional perturbation applied to one of those experiments which causes the blue line to start at negative values.
    \item CanESM has net deforrestation by the end of the century, but increases in cVeg. It also has no significant change in global temperature by the end of the century.
    \item ACCESS-ESM is roughly in the middle of the model spread. More similar to MPI-ESM and UKESM than other models.
\end{itemize}

\begin{figure}[H]
    \centering
    \includegraphics[width=\linewidth]{../plots/tas_trends.png}
    \caption{Surface air temperature trends in the difference between the forestation scenario and the reference simulaiton. Solid lines indicate trends that are statistically significant at the 5\% level, and dotted lines are not significant trends. '+' and '-' denote the sign of the trend.}
    \label{fig:models_tas_trends}
\end{figure}

\subsection{ACCESS-ESM1.5}

\subsubsection{Changes to the Global Carbon Cycle}

\begin{figure}[H]
    \centering
    \begin{subfigure}[b]{0.4\linewidth}
        \includegraphics[width=\linewidth]{../plots/CO2_atm_ppm_diff.png}
    \end{subfigure}
    \begin{subfigure}[b]{0.4\linewidth}
        \includegraphics[width=\linewidth]{../plots/esm-ssp585-ssp126Lu_budget.png}
    \end{subfigure}
    \caption{Top left: Difference in atmospheric CO$_2$ concentration between the forestation scenario and ssp585. Shading indicates the ensemble range. Top right: ensemble mean of carbon flux budget in the forestation scenario.}
    \label{fig:atmosphere_carbon}
\end{figure}

\begin{figure}[H]
    \centering
    \begin{subfigure}[b]{0.4\linewidth}
        \includegraphics[width=\linewidth]{../plots/fgco2_ocean_carbon_aff_ssp585_flux.png}
    \end{subfigure}
    \begin{subfigure}[b]{0.4\linewidth}
        \includegraphics[width=\linewidth]{../plots/fgco2_ocean_carbon_aff_ssp585_cumulative.png}
    \end{subfigure}
    \caption{Left: Difference between the forestation scenario and ssp585 in downward ocean carbon mass flux. Right: Difference between the forestation scenario and ssp585 for the cumulative downward ocean carbon mass flux. Shading indicates the ensemble range.}
    \label{fig:ocean_carbon}
\end{figure}

\begin{table}[H]
    \centering
    \begin{tabular}{@{}llll@{}}
    \hline
        & $\Delta$ Carbon (Pg) & \% present day & \% of 2080-2100 ssp585 \\ \hline
cLand   & 22.0±5.3    & 1.3±0.3        & 1.3±0.3                \\
cVeg    & 24.8±2.7    & 3.7±0.4        & 3.3±0.4                \\
cLitter & 0.4±0.3     & 0.8±0.6        & 0.9±0.7                \\
cSoil   & -4.3±2.3    & -0.5±0.3       & -0.5±0.3               \\ \hline
    \end{tabular}
    \caption{Cpools expressed as percentage of the present day carbon content or the end of ssp585}
    \label{tab:cpools_table}
\end{table}

\subsubsection{Climate}

\begin{itemize}
    \item Climate warms about 4 \textcelsius{} by 2100 and precipitation increases by 0.25e-5 kg m-2 s-1 \footnote{Need to change the units of this to something more human readable}
    \item There isn't much difference between the forestation scenario and ssp585.
\end{itemize}

\begin{figure}[H]
    \centering
    \begin{subfigure}[b]{0.45\linewidth}
        \includegraphics[width=\linewidth]{../plots/tas_ACCESS-ESM1.5_esm-ssp585_ensembles.png}
    \end{subfigure}
    \begin{subfigure}[b]{0.45\linewidth}
        \includegraphics[width=\linewidth]{../plots/tas_ACCESS-ESM1.5_esm-ssp585_ensembles_diff.png}
    \end{subfigure}
    \begin{subfigure}[b]{0.45\linewidth}
        \includegraphics[width=\linewidth]{../plots/pr_ACCESS-ESM1.5_esm-ssp585_ensembles.png}
    \end{subfigure}
    \begin{subfigure}[b]{0.45\linewidth}
        \includegraphics[width=\linewidth]{../plots/pr_ACCESS-ESM1.5_esm-ssp585_ensembles_diff.png}
    \end{subfigure}
    \caption{Climate in esm-ssp585-ssp126Lu. Left: Absolute values from the forestation simulation (esm-ssp585-ssp126Lu). Right: difference between esm-ssp585-ssp126Lu simulation and esm-ssp585}
    \label{fig:climate}
\end{figure}

\subsubsection{Regional/local changes in vegetation and extremes}

\begin{itemize}
    \item There is an increase in vegetation carbon of about 25 Pg C by the end of the century.
    \item The increased CO$_2$ draw-down from the combined effect of forestation and CO$_2$ fertilization seems to be offset by the increase in respiration in the litter and soil by the end of the century.
    \item cSoil is lower than SSP585 by about 4 Pg carbon.
\end{itemize}

\begin{figure}[H]
    \centering
    \begin{subfigure}[b]{0.4\linewidth}
        \includegraphics[width=\linewidth]{../plots/cVeg_ACCESS-ESM1.5_esm-ssp585-ssp126Lu_ensembles_anomalies.png}
    \end{subfigure}
    \begin{subfigure}[b]{0.4\linewidth}
        \includegraphics[width=\linewidth]{../plots/cVeg_ACCESS-ESM1.5_esm-ssp585-ssp126Lu_ensembles_diff.png}
    \end{subfigure}
    \begin{subfigure}[b]{0.4\linewidth}
        \includegraphics[width=\linewidth]{../plots/cLitter_ACCESS-ESM1.5_esm-ssp585-ssp126Lu_ensembles_anomalies.png}
    \end{subfigure}
    \begin{subfigure}[b]{0.4\linewidth}
        \includegraphics[width=\linewidth]{../plots/cLitter_ACCESS-ESM1.5_esm-ssp585-ssp126Lu_ensembles_diff.png}
    \end{subfigure}
    \begin{subfigure}[b]{0.4\linewidth}
        \includegraphics[width=\linewidth]{../plots/cSoil_ACCESS-ESM1.5_esm-ssp585-ssp126Lu_ensembles_anomalies.png}
    \end{subfigure}
    \begin{subfigure}[b]{0.4\linewidth}
        \includegraphics[width=\linewidth]{../plots/cSoil_ACCESS-ESM1.5_esm-ssp585-ssp126Lu_ensembles_diff.png}
    \end{subfigure}
\begin{subfigure}[b]{0.4\linewidth}
        \includegraphics[width=\linewidth]{../plots/cLand_ACCESS-ESM1.5_esm-ssp585-ssp126Lu_ensembles_anomalies.png}
    \end{subfigure}
    \begin{subfigure}[b]{0.4\linewidth}
        \includegraphics[width=\linewidth]{../plots/cLand_ACCESS-ESM1.5_esm-ssp585-ssp126Lu_ensembles_diff.png}
    \end{subfigure}
    \caption{Carbon pools. Left: esm-ssp585-ssp126Lu relative to the 2005–2025 period. Right: difference between the esm-ssp585-ssp126Lu and esm-ssp585.}
    \label{fig:cpools}
\end{figure}

\begin{figure}[H]
    \centering
    \begin{subfigure}[b]{0.4\linewidth}
        \includegraphics[width=\linewidth]{../plots/gpp_ACCESS-ESM1.5_esm-ssp585-ssp126Lu_ensembles_anomalies.png}
    \end{subfigure}
    \begin{subfigure}[b]{0.4\linewidth}
        \includegraphics[width=\linewidth]{../plots/gpp_ACCESS-ESM1.5_esm-ssp585-ssp126Lu_ensembles_diff.png}
    \end{subfigure}
    \begin{subfigure}[b]{0.4\linewidth}
        \includegraphics[width=\linewidth]{../plots/npp_ACCESS-ESM1.5_esm-ssp585-ssp126Lu_ensembles_anomalies.png}
    \end{subfigure}
    \begin{subfigure}[b]{0.4\linewidth}
        \includegraphics[width=\linewidth]{../plots/npp_ACCESS-ESM1.5_esm-ssp585-ssp126Lu_ensembles_diff.png}
    \end{subfigure}
    \begin{subfigure}[b]{0.4\linewidth}
        \includegraphics[width=\linewidth]{../plots/ra_ACCESS-ESM1.5_esm-ssp585-ssp126Lu_ensembles_anomalies.png}
    \end{subfigure}
    \begin{subfigure}[b]{0.4\linewidth}
        \includegraphics[width=\linewidth]{../plots/ra_ACCESS-ESM1.5_esm-ssp585-ssp126Lu_ensembles_diff.png}
    \end{subfigure}
    \begin{subfigure}[b]{0.4\linewidth}
        \includegraphics[width=\linewidth]{../plots/rh_ACCESS-ESM1.5_esm-ssp585-ssp126Lu_ensembles_anomalies.png}
    \end{subfigure}
    \begin{subfigure}[b]{0.4\linewidth}
        \includegraphics[width=\linewidth]{../plots/rh_ACCESS-ESM1.5_esm-ssp585-ssp126Lu_ensembles_diff.png}
    \end{subfigure}
    \begin{subfigure}[b]{0.4\linewidth}
        \includegraphics[width=\linewidth]{../plots/nbp_ACCESS-ESM1.5_esm-ssp585-ssp126Lu_ensembles_anomalies.png}
    \end{subfigure}
    \begin{subfigure}[b]{0.4\linewidth}
        \includegraphics[width=\linewidth]{../plots/nbp_ACCESS-ESM1.5_esm-ssp585-ssp126Lu_ensembles_diff.png}
    \end{subfigure}
    \caption{Carbon fluxes in esm-ssp585-ssp126Lu.  Left: relative to 2005–2025. Right: relative to ssp-585.}
    \label{fig:cflux}
\end{figure}

\begin{itemize}
    \item Australian cVeg increases by ~0.25 PgC in the forestation scenario. But the difference from the reference simulation is 0.
\end{itemize}

\begin{figure}[H]
    \centering
    \begin{subfigure}[b]{0.4\linewidth}
        \includegraphics[width=\linewidth]{../plots/cVeg_australia_anom.png}
    \end{subfigure}
    \begin{subfigure}[b]{0.4\linewidth}
        \includegraphics[width=\linewidth]{../plots/cVeg_australia_diff.png}
    \end{subfigure}
    \begin{subfigure}[b]{0.4\linewidth}
        \includegraphics[width=\linewidth]{../plots/cLitter_australia_anom.png}
    \end{subfigure}
    \begin{subfigure}[b]{0.4\linewidth}
        \includegraphics[width=\linewidth]{../plots/cLitter_australia_diff.png}
    \end{subfigure}
    \begin{subfigure}[b]{0.4\linewidth}
        \includegraphics[width=\linewidth]{../plots/cSoil_australia_anom.png}
    \end{subfigure}
    \begin{subfigure}[b]{0.4\linewidth}
        \includegraphics[width=\linewidth]{../plots/cSoil_australia_diff.png}
    \end{subfigure}
    \caption{Left: Australian region absolute carbon stocks. Right: Australian region  difference between forestation scenario and ssp585.}
    \label{fig:aus_region}
\end{figure}

\begin{itemize}
    \item The esm-ssp585-ssp126Lu land use has bot forestation and deforestation.
    \item Forestation occurs in Boreal Eurasia, Boreal North America and Brazil.
    \item Forestation and deforestation occurs in eastern North America, Central Africa and East Asia.
    \item Deforestation occurs in western Eurasia.
    \item Low forestation/deforestation occurs in Australia.
    \item No appreciable difference in climate variables.
    \item Increases in cVeg for all regions except Boreal North America, where some ensemble members show decreases in cVeg by the end of the century.
    \item Decreases in cSoil for Australia, Amazonia, Eastern North America. Western Eurasia, East Asia, Boreal Eurasia, Boreal North America, generally also show decreases in cSoil but there are a few ensemble members that show increases by the end of the century.
\end{itemize}

\begin{figure}[H]
    \centering
    \begin{subfigure}[b]{0.4\linewidth}
        \includegraphics[width=\linewidth]{../plots/cLand_amazonia_diff.png}
    \end{subfigure}

    \begin{subfigure}[b]{0.4\linewidth}
        \includegraphics[width=\linewidth]{../plots/cLand_easternnorthamerica_diff.png}
    \end{subfigure}
    \begin{subfigure}[b]{0.4\linewidth}
        \includegraphics[width=\linewidth]{../plots/cLand_eastasia_diff.png}
    \end{subfigure}
    \begin{subfigure}[b]{0.4\linewidth}
        \includegraphics[width=\linewidth]{../plots/cLand_borealnorthamerica_diff.png}
    \end{subfigure}
    \begin{subfigure}[b]{0.4\linewidth}
        \includegraphics[width=\linewidth]{../plots/cLand_centralafrica_diff.png}
    \end{subfigure}
    \caption{Experiment differences for each region in \ref{fig:box_regions}}
    \label{fig:aus_region_cveg_tas}
\end{figure}

\begin{itemize}
    \item The lack of climate warming mitigation potential in the forestation scenario may be due to redistribution of the global carbon budget into the natural sinks.
        An increase in vegetation uptake would lead to a decrease in atmospheric CO$_2$ accumulation, which would result in a reduction of partial CO$_2$ pressure on the ocean and hence lead to a reduction of CO$_2$ in the ocean sink.
    \item The LUH2 data set severely underestimates the tree cover that is originally dictated by IMAGE (the integrated assessment model that produces the SSP1-2.6 scenario).
    \item The bio-geophysical impacts of growing trees causes localized warming at the extreme ends of the temperature distribution, especial in the higher latitudes where trees tend to be slow growing and the bare ground either has a high albedo or is covered in snow. This warming might offset the climate mitigation potential of forestation.
\end{itemize}

\section{Conclusion}

\begin{itemize}
    \item Likely no impact of forestation on global climate under a high emissions and high global warming scenario. Climate is dominated by the warming signal.
    \item Evidence of some increase in local scale warm extremes.
    \item Evidence in a shift in the global carbon balance; increased uptake of carbon on the land of ~25 PgC by 2100 and a decrease in the uptake of carbon by the ocean.
\end{itemize}

\printbibliography

\section{Supplementary Material}
\setcounter{figure}{0}

\subsection{Atmosphere}
\begin{figure}[H]
    \centering
    \begin{subfigure}[b]{0.4\linewidth}
        \includegraphics[width=\linewidth]{../plots/CO2_aff_and_ssp585.png}
    \end{subfigure}
    \caption{Atmospheric CO2 concentration for the model ACCESS-ESM1-5.}
    \label{fig:ACCESS_CO2_conc}
\end{figure}

\subsection{PFTs}

\begin{figure}[H]
    \centering
    \begin{subfigure}[b]{0.4\linewidth}
        \includegraphics[width=\linewidth]{../plots/treeFrac_esm-ssp585_anomaly.png}
    \end{subfigure}
    \begin{subfigure}[b]{0.4\linewidth}
        \includegraphics[width=\linewidth]{../plots/grassFrac_esm-ssp585_anomaly.png}
    \end{subfigure}
    \begin{subfigure}[b]{0.4\linewidth}
        \includegraphics[width=\linewidth]{../plots/cropFrac_esm-ssp585_anomaly.png}
    \end{subfigure}
    \caption{Difference in area of trees, grass and crop between 2015 and 2100 for esm-ssp585 in ACCESS-ESM1.5}
    \label{fig:ACCESS_land_cover}
\end{figure}

\begin{figure}[H]
    \centering
    \begin{subfigure}[b]{0.45\linewidth}
        \includegraphics[width=\linewidth]{../plots/CABLE_forests_deciduous_needle.png}
    \end{subfigure}
    \caption{Global change in deciduous needleleaf forest in ACCESS-ESM1-5.}
    \label{fig:ACCESS_dec_needle_cover}
\end{figure}

\subsection{Model intercomparison}

\begin{figure}[H]
    \centering
    \includegraphics[width=\linewidth]{../plots/pr_trends.png}
    \caption{Surface global precipitation rate trends in the difference between the forestation scenario and the reference simulaiton. Solid lines indicate trends that are statistically significant at the 5\% level, and dotted lines are not significant trends. '+' and '-' denote the sign of the trend.}
    \label{fig:models_pr_trends}
\end{figure}

\begin{figure}[H]
    \centering
    \begin{subfigure}[b]{0.45\linewidth}
        \includegraphics[width=\linewidth]{../plots/cLand_model_intercomparison_esm-ssp585-ssp126Lu.png}
    \end{subfigure}
    \begin{subfigure}[b]{0.45\linewidth}
        \includegraphics[width=\linewidth]{../plots/cVeg_model_intercomparison_esm-ssp585-ssp126Lu.png}
    \end{subfigure}
    \begin{subfigure}[b]{0.45\linewidth}
        \includegraphics[width=\linewidth]{../plots/cLitter_model_intercomparison_esm-ssp585-ssp126Lu.png}
    \end{subfigure}
    \begin{subfigure}[b]{0.45\linewidth}
        \includegraphics[width=\linewidth]{../plots/cSoil_model_intercomparison_esm-ssp585-ssp126Lu.png}
    \end{subfigure}
    \begin{subfigure}[b]{0.45\linewidth}
        \includegraphics[width=\linewidth]{../plots/tas_model_intercomparison_esm-ssp585-ssp126Lu.png}
    \end{subfigure}
    \begin{subfigure}[b]{0.45\linewidth}
        \includegraphics[width=\linewidth]{../plots/pr_model_intercomparison_esm-ssp585-ssp126Lu.png}
    \end{subfigure}
    \caption{Absolute values of the global mean carbon pools, 2m surface temperature and precipitation for each model. The solid line for ACCESS ESM-1-5 is the ensemble mean and the shading indicates the ensemble range.}
    \label{fig:models_absolute}
\end{figure}

\subsection{Regional analysis}

\begin{figure}[H]
    \centering
    \begin{subfigure}[b]{0.4\linewidth}
        \includegraphics[width=\linewidth]{../plots/tas_amazonia_diff.png}
    \end{subfigure}
    \begin{subfigure}[b]{0.4\linewidth}
        \includegraphics[width=\linewidth]{../plots/tas_easternnorthamerica_diff.png}
    \end{subfigure}
    \begin{subfigure}[b]{0.4\linewidth}
        \includegraphics[width=\linewidth]{../plots/tas_eastasia_diff.png}
    \end{subfigure}
    \begin{subfigure}[b]{0.4\linewidth}
        \includegraphics[width=\linewidth]{../plots/tas_borealnorthamerica_diff.png}
    \end{subfigure}
    \begin{subfigure}[b]{0.4\linewidth}
        \includegraphics[width=\linewidth]{../plots/tas_centralafrica_diff.png}
    \end{subfigure}
    \caption{Mean surface air temperature for each region in ACCESS-ESM1-5.}
    \label{fig:ACCESS_tas_regions}
\end{figure}

\end{document}

