% Draft manuscript on forestation in CMIP6 to be submitted to GBC or similar journal.
% Note: Figures are plotted using .png files for now since SVG rendering is very
% slow for LaTeX. At a later date I will change the figures to .svg files. For example:
%\includesvg[width=\linewidth]{../plots/cVeg_aus_anom.svg}
% Journal candidates:
% - Earth system dynamics
% Preamble.
\documentclass[]{article}
\usepackage[a4paper, total={170mm,257mm}, left=20mm, top=20mm,]{geometry}
\usepackage{svg} % Use SVGs only for final draft.
\usepackage{float} % Allows the [H] option meaning "place figure here and only here"
\usepackage[section]{placeins} % Defines /FloatBarrier so that figures can't move.
\usepackage{subcaption} % Allows subfigures and captions
\usepackage{lineno} % Enable line numbers.
\usepackage{textcomp}
\usepackage[style=apa]{biblatex}
\usepackage{hyperref}
\usepackage{amsmath}
\addbibresource{afforestation_references.bib}
%\setlength{\parindent}{1cm}
\linenumbers
\title{Limited mitigation potential of forestation under a high emissions scenario: multi-model and single model ensembles}
\author{Tammas Loughran, Tilo Ziehn, Rachel Law, Spencer Liddicoat, Josep Canadell, Julia Pongratz, David Lawrence, Akihiko Ito}

\begin{document}

\maketitle

\begin{center}
    \Large
    \vspace{0.9cm}
    \textbf{Abstract}
\end{center}

Forestation is a major component of future long-term emissions reduction and CO$_2$ removal strategies, but the viability of carbon stored in vegetation under future climates is highly uncertain.
We analyze the results from 7 CMIP6 models of a scenario with high fossil fuel emissions and forest expansion.
This scenario demonstrates the ability of forestation strategies to mitigate climate change under continued increasing CO$_2$ emissions and includes the potential impacts of increased CO$_2$ concentration and a warming climate on vegetation growth.
The model intercomparison shows that moderate forestation (under scenario SSP1-2.6)  as a CO$_2$ removal strategy has limited impact on global climate under a high global warming scenario (SSP5-8.5), despite generating a substantial long-term carbon sink of 10--60 Pg C by 2100.
By using a single-model initial condition ensemble, we also show that there may be local increases in warm extremes in response to forestation, particularly for decreases in the number of cool days.
Furthermore, we find evidence of shift in the global carbon balance, whereby increased uptake of carbon on the land of $\sim$25 Pg C by 2100 has a concomitant decrease in the uptake of carbon by the ocean due to reduced atmospheric CO$_2$ concentrations.

\raggedright
\parindent=.35in % Setting raggedright removes paragraph indents. This puts them back.

\section{Introduction}

Forests cover approximately 31\% of the global land surface \parencite{luyssaert_land_2014,fao_global_2020} and are responsible for the removal of 30\% of total anthropogenic emissions from the atmosphere \parencite{friedlingstein_global_2022}.
Forestation is therefore often thought of as a viable strategy to remove CO$_2$ from the atmosphere and mitigate global warming \parencite{house_maximum_2002, griscom_natural_2017, smith_long-term_2022}.
Most decarbonization pathways to limit global warming to below 1.5 or 2 \textcelsius{}, consistent with the Paris Agreement, require not only a reduction of fossil fuel emissions, but also CO$_2$ removal to offset industrial and agricultural emissions that are difficult to abate \parencite{babiker_crosssectoral_2022}.
The most commonly used practice to remove CO2 from the atmosphere in decarbonization pathways is forestation that includes a) reforestation: forest regrowth in abandoned agricultural and pasture lands, and direct tree planting, and b) afforestation: tree planting in areas not previously forested.

Forestation and deforestation affect the climate in two main ways \parencite{pongratz_biogeophysical_2010,ito_biogeophysical_2020,zhu_comparable_2023}.
Firstly, by biogeochemical effects, i.e., changes to the global carbon cycle and carbon storage pools that affect atmospheric CO$_2$ concentration and, therefore, the radiative absorption of the atmosphere.
And secondly by biogeophysical effects, i.e., changes in the physical properties of the land surface such as albedo, roughness and evapotranspiration efficiency, which in turn influence the surface energy balance \parencite{betts_offset_2000,bala_combined_2007,winckler_importance_2019}.
In general, forestation causes a global cooling biogeochemical effect as carbon is taken from the atmosphere and stored in vegetation and soils.
However, the biogeophysical impacts of forestation are more varied, with the effects of albedo and roughness having opposing impacts that might dominate more or less at different latitudes.

Historically, there has been substantial deforestation in temperate forests of Eurasia and North America and in the last few decades deforestation has been focused on the tropics (\cite{goldewijk_estimating_2001}).
The net effect of deforestation is to cool the climate globally due to an increase in albedo (\cite{davin_climatic_2010}).
While the albedo-induced cooling is a result of the changes in the global planetary energy balance, reinforced by the ocean, the biogeophysical effects at the site of the deforestation are generally a warming effect \parencite{winckler_nonlocal_2019}: Locally, the reduction in absorbed energy is compensated for by a reduction in turbulent heat fluxes \parencite{winckler_importance_2019}.
As a result, the albedo changes have a minor influence on local temperatures.
Instead, the reduction in roughness transforming forest to short, smooth grass or cropland vegetation leads to less efficient transfer of heat from the surface into the atmosphere, which induces a warming both in the annual mean and daily and seasonal warm extremes \parencite{winckler_different_2019}; model results are confirmed by observation-based estimates (which by way of their setup capture only local effects;  \parencite{alkama_biophysical_2016,bright_local_2017} see \cite{pongratz_land_2021} for a review of the climatic effects of forest cover changes from local to global scale).
In another modeling study of deforestation, \cite{boysen_global_2020} show a cooling of -0.22$\pm$0.2 \textcelsius{} among 9 climate models in idealized deforestation simulations with constant atmospheric CO$_2$ concentration; in the tropics, the warming effect of local surface property changes dominates over the global cooling signal in most models.
\cite{hong_impacts_2022} also show that under a future deforestation scenario this cooling effect might reduce the incidence of hot extremes by 5.5--0.9\%.
However, the effect of future forestation may not necessarily be merely the inverse of the effect of future deforestation.

There have been several of previous studies on the potential of forestation to remove CO$_2$ from the atmosphere, each using different methods of quantification.
For example, early studies like \cite{house_maximum_2002} approximated the maximum hypothetical potential by reversing all historical forest loss by 40-70 ppm by 2100. Similarly, \cite{lenton_radiative_2009} used a simplified ``back-of-the-envelope" analytical calculation approach to estimate the radiative forcing effect of forestation, finding that it has substantial potential relative to most other geoengineering methods.
They estimate a removal of 73 Pg C by forestation can result in a decrease of 0.37 W m$^{-2}$ radiative forcing by 2100.
However, besides assuming hypothetical scenarios of forestation, such simplified estimates largely disregard impacts of future environmental changes on forests and assume that the land carbon storage is stable on long time scales, the decay of which under a future climate could only be estimated with an earth system model.

More sophisticated modeling studies better represent the complexity of the net effects of biogeophysical and biogeochemical impacts and their dynamics depending on future environmental changes.
For example, \cite{sonntag_reforestation_2016} quantified the CO$_2$ removal potential using an Earth System Model, in which a high CO$_2$ emissions scenario is simulated (taken from RCP8.5) in combination with the ``forestation" land-use from a low emissions scenario (taken from RCP4.5).
This study used the concentration driven CMIP5 (Coupled Model Intercomparison Project Phase 5) version of the MPI-ESM and the CMIP5 representative concentration pathways.
They find that a decrease of about 85 ppm (215 Pg C) in atmospheric CO$_2$ concentrations by 2100 from forestation results in a cooling of 0.27 K globally, with a dampening of heat extremes through biogeophysical effects in some densely populated regions (also reiterated in \cite{sonntag_quantifying_2018}).

Some other experimental designs kept to idealized assumptions and represent a more extreme deployment of forestation, where a a large and potentially unfeasible portion of non-forested lands are forested.
\cite{de_hertog_biogeophysical_2022}, for example, conducted an experiment with all non-forested lands, except bare ground, are forested in a checker-board pattern.
While this is not meant as a real-world application, it allows for diagnosis of local and non-local effects of forestation, demonstrating that in those models the local biogeophysical effects from forestation produce a cooling in the tropics, while the non-local effects result in a large-scale warming.

To date, a multi-model intercomparison of the potential of forestation to remove CO$_2$ and mitigate climate change under a realistic scenario is absent from the literature.
This study aims to quantify the CO$_2$ removal potential of a high-emissions forestation scenario (SSP5-8.5) using CMIP6 models in an interactive emissions-driven carbon cycle configuration.
We will assess the viability of the carbon stores, and the biogeochemical and biogeophysical impacts of forestation on the climate.
The CMIP6 ensemble of models provides uncertainty related to model structure and parameters.
Furthermore, there is an ensemble of simulations available for a single model for estimation of uncertainty arising from internal climate variability. 

\section{Methods}

\subsection{Experiments}

The CMIP6 experimental design specifies a set of standard simulations called the Diagnostic, Evaluation and Characterization of Klima (DECK) \parencite{eyring_overview_2016}, which are typically used as the foundation to study more specific research questions.
The DECK includes a pre-industrial control simulation (\textit{piControl}) with constant greenhouse gas concentration forcing, and a historical (\textit{historical}) simulation with transient historical greenhouse gas concentration forcing.
Furthermore, the DECK specifies corresponding simulations (\textit{esm-piControl} and \textit{esm-hist}) that are run in ``Earth system model" mode (i.e. there is a fully interactive dynamical carbon cycle) with fossil fuel and industrial greenhouse gas emission forcing during the historical period.
In order to capture the carbon cycle and dynamics of the entire Earth system in response to forestation, including biogeochemical and biogeophysical feedbacks of the land surface and climate system, we use only the Earth system model simulations from relevant intercomparison projects. These emissions-driven simulations are initialized at 2015 from the end of the \textit{esm-hist} experiment.

The forestation scenario we use is the Land Use Model Intercomparison Project's \textit{esm-ssp585-ssp126Lu}, spanning the years 2015--2100 \parencite{lawrence_land_2016}.
This simulation features the surface fossil fuel related CO$_{2}$ emissions from the SSP5-8.5 scenario, and the land-use change from the SSP1-2.6 scenario \parencite{oneill_scenario_2016}.
The SSP5-8.5 scenario emissions represent a high emissions future pathway that results in a radiative forcing of 8.5 W m$^{-2}$ in 2100.
The SSP1-2.6 scenario land-use change assumes a future of sustainable development \parencite{van_vuuren_energy_2017}.
This means low population growth, low pressures on land-use, environmentally friendly changes in diets and increased agricultural efficiency and yields, resulting in abandonment of agricultural lands.
This allows for the expansion of natural lands and forest cover.
We use this scenario because it represents a plausible future scenario that would provide a lower bound on the survivability of vegetation and CO$_2$ draw-down potential of the land surface under a warmer climate.

The reference simulation we use to compare to \textit{esm-ssp585-ssp126Lu} is the \textit{esm-ssp585} from the Coupled Climate-Carbon Cycle Model Intercomparison Project (C4MIP) \parencite{jones_c4mip_2016}.
This is the emissions-driven Earth system model simulation corresponding to the SSP5-8.5 high greenhouse gas scenario that assumes that development is driven by fossil fuels \parencite{oneill_scenario_2016}.
In general, the land use change in this scenario has less expansion of forested lands than \textit{esm-ssp585-ssp126Lu}, but is a mixture of moderate forestation and deforestation.
The notable difference in land-use change compared to the forestation scenario is the deforestation in the central African region (Figure S\ref{fig:land_use}).
By taking the difference between the forestation simulation and the \textit{esm-ssp585} simulation for any variable X (Equation \ref{equ:diff}), we can examine the impact of forestation on the climate and the carbon cycle.

\begin{equation}
    X|_{for} = X|_{esm-ssp585-ssp126Lu} - X|_{esm-ssp585}
    \label{equ:diff}
\end{equation}

The land-use change data set used in all CMIP6 experiments is the Land-use Harmonization data set version 2 (LUH2; \cite{hurtt_harmonization_2020}), which is translated to model plant functional types by each model (for example \cite{di_vittorio_land_2014}).
\textit{esm-ssp585-ssp126Lu} is typically characterized with forestation, while the \textit{esm-ssp585} has a mixture of deforestation and forestation.
There are only a few areas where there is deforestation in the \textit{esm-ssp585-ssp126Lu} experiment.
These occur in deciduous broad leaf forests in eastern North America, China and western Russia (Figure \ref{fig:land_use_map} is of primary and secondary forests changes).
Furthermore, it is important to acknowledge that the LUH2 SSP1-2.6 scenario underestimates the tree cover that is originally dictated by IMAGE (the integrated assessment model that produces the SSP1-2.6 scenario).
This is due to differences in the definition of tree cover in the integrated assessment models as well as the effects of harmonization of that data with observed historical cover fractions.
Despite LUH2 providing additional forest cover data to match the forestation in IMAGE, only CESM utilized it of all the models in this study.
Therefore we consider this scenario as having moderate forestation and would represent a lower end of the future mitigation potential.

\begin{figure}[H]
    \centering
    \begin{subfigure}[b]{\linewidth}
        \includegraphics[width=\linewidth]{../plots/Figure_1.png}
    \end{subfigure}
%    \begin{subfigure}[b]{\linewidth}
%        \includegraphics[width=\linewidth]{../plots/tree_frac_deltat_ssp585.png}
%    \end{subfigure}
    \caption{Changes in tree fraction from the Land-use Harmonization 2 for (a) the SSP1-2.6 scenario and (b) the SSP5-8.5 scenario between the year 2100 and 2015. Boxes demark the areas used in the ACCESS-ESM1-5 regional analysis. The black crosses mark the locations of large forestation changes that are used for daily temperature distributions.}
    \label{fig:land_use}
\end{figure}

\subsection{Participating models}

There are 7 Earth system models that participated in both LUMIP and C4MIP with simulations available for \textit{esm-ssp585-ssp126Lu} and \textit{esm-ssp585}.
NorESM5 contributed a simulation to LUMIP for the \textit{esm-ssp585-ssp126Lu} experiment, however we have excluded it since it was run in concentration driven mode.
We also excluded BCC-CSM2-MR upon request of the developers due to a bug in the soil respiration.
A brief overview of these models is presented in Table \ref{tab:models}.
2 models included wildfire schemes, 2 models included dynamical vegetation (i.e., the biogeographical distribution of the vegetation types changes in response to environmental changes), and 6 models included nitrogen nutrient limitation.
ACCESS-ESM1-5 is the only model to include phosphorus limitation.
Furthermore, ACCESS-ESM1-5 is the only model to have multiple ensemble members available for both simulations, of which there are 10 for each experiment.

CanESM5 has different ensemble initialization methods for \textit{esm-ssp585} and \textit{esm-ssp585-ssp126Lu}.
The former uses r1i1p1f1 and the latter uses r1i1p2f1, which features recent bug-fixes in the model.
To account for the slightly different initial conditions, the CanESM5 model (green line in \ref{fig:models_cpools}) has been bias corrected by subtracting the difference in the carbon pools between the reference and forestation simulations at the start of the experiment (2015).
This makes the CanESM5 comparable with other models for all variables.

\begin{table}[H]
    \centering
    \begin{tabular}{lllllllll}
        \hline
Model         & Fire & N   & P   & Dyn.  & PFTs     & Natural & Res.           & Reference                           \\ \hline
ACCESS-ESM1-5 & No   & Yes & Yes & No    & 10       & 9       & 1.8758×1.258°   & \cite{ziehn_australian_2020}   \\
%BCC-CSM2-MR   & No   & Yes & No  & No    & 14       & 11      & 110×110km       & \cite{li_development_2019}     \\
CanESM5       & No   & No  & No  & No*   & 9        & 7       & 2.8×2.8°  & \cite{swart_canadian_2019}     \\
GFDL-ESM4     & Yes  & No  & No  & No    & 7        & 5       & 100×100km       & \cite{dunne_gfdl_2020}         \\
MIROC-ES2L    & No   & Yes & No  & No    & 14       & 13      & 2.8×2.8°        & \cite{hajima_development_2020} \\
MPI-ESM1-2-LR & Yes  & Yes & No  & Yes   & 12       & 8      & 1.8° ×1.8°       & \cite{mauritsen_developments_2019}  \\
%NorESM2-LM    & Yes  & Yes & No  & No    & 22       & 14      & 2×2°            & \cite{seland_norwegian_2020}   \\
CESM2         & Yes  & Yes & No  & No    & 22       & 14      & 1.25°×0.9°      & \cite{danabasoglu_community_2020} \\
UKESM1-0-LL   & No   & Yes & No  & Yes   & 12       & 8       & 1.25×1.875°     & \cite{sellar_ukesm1_2019}      \\ \hline
    \end{tabular}
    \caption{Models participating in both Land-Use Model Intercomparison Project and Coupled Climate-Carbon model Intercomparison Project with available simulations for \textit{esm-ssp585-ssp126Lu} and \textit{esm-ssp585}. The columns indicate whether the models land surface component have representations of wildfire, nutrients, dynamical vegetation and the number of plant functional types. Natural plant functional types exclude agricultural crops and pasture. *Can-ESM5 does have plant dynamics in the sense of plant demographics but not dynamical plant type competition.}
    \label{tab:models}
\end{table}

\subsection{Data}

The data used in this study are available from the Earth System Grid Federation.
For each model, we use the following monthly mean variables: tree cover fraction (treeFrac), vegetation, litter, soil and total land carbon (cVeg, cLitter, cSoil, cLand respectively), atmospheric CO$_2$ concentrations (co2), ocean CO$_2$ flux (fgco2), 1.5 m surface air temperature (tas) and precipitation rate (pr).
From ACCESS-ESM1.5 we also use daily maximum temperatures (tasmax).
Some variables are not available for particular models, such as treeFrac data from MIROC, and atmospheric CO$_2$ concentration from CanESM and GFDL-ESM4.

Changes in the treeFrac variable typically represents the changes in forests, but definition of forest cover can vary.
ESMs represent forest area as a fraction of a grid cell's land surface rather than crown cover, which is an important distinction since definitions of forests vary greatly with crown cover \parencite{zomer_land_2008}.
Nevertheless, the forest and tree definition can still differ between models depending on how the LUH2 forcing data are translated into model PFTs.
For example, CanESM5 and GFDL-ESM4 do not have an explicit representations of shrubs but considers them as tree PFTs and therefore areas otherwise considered as shrubs in other models would be included in treeFrac.
Also none of the models here have a representation of rangelands and thus the LUH2 rangelands can be variously interpreted by the models as forest, pasture, shrublands or savanna, which may or may not be considered as a woody tree PFTs.

\subsection{Analysis and statistical methods}

For the global analysis across all models, we calculate trends in the difference of global mean temperatures and precipitation to examine the change in temperature as forests expand.
For this we use the Theil-Sen slope estimator and test its significance at the 5\% level using the Mann-Kendall trend test.

For a more detailed analysis of regional carbon uptakes and temperature impacts of forestation, the ACCESS-ESM1-5 regional analyses have been divided into 6 regions that feature notable changes in tree cover.
These regions are shown in Figure \ref{fig:land_use}.
The temperature and precipitation values over the ocean are masked out to represent changes only over the land surface.

Much of the tree cover changes between the two simulations occurs in a handful of concentrated regions.
To examine the local scale impact of dramatic forestation on temperature and precipitation, histograms of the frequency distributions for specific grid-points are calculated for the locations shown by the crosses in Fig. \ref{fig:land_use}.
These large forestation regions are in Eastern North America, East Asia and Amazonia.
For these, we focus only on ACCESS-ESM1-5, since it is the only model that contributed 10 ensemble members, and thus the statistics of the distribution are most robust for this model.
For the grid point maximum temperature frequency histograms, the difference in the temperature distribution in the two simulations in response to forestation is tested using the 2-sample Kolmogorov-Smirnov test for the equality of distributions at the 5\% level.

Lastly, to examine the relationship of forestation on climate, correlations were done on ensemble mean surface air temperature and tree fraction using the Spearman's rank correlation and the significance was tested at the 5\% level \parencite{kokoska2000crc}.

\section{Results and discussion}

\subsection{Global multi-model inter-comparison}

\subsubsection{Carbon cycle}

Each model has a unique internal representation of woody tree biomes and pre-industrial potential vegetation.
Therefore, each model interprets the allocation of natural lands from LUH2 to model-specific plant functional types (PFTs) differently.
Combined with differences in grid resolution, this results in a variety of changes in global tree cover for each model. This is demonstrated in Fig. \ref{fig:land_use_map}, which shows the difference in tree cover fraction between the two scenarios for each model at 2100 (Fig. S\ref{fig:tree_area_maps_ssp126Lu} also shows the temporal change for each experiment).
UKESM1-0-LL, CanESM5 and MPI-ESM1-2-LR include vegetation dynamics that respond to changes in climate, and these are the models that deviate from the LUH2 land-use forcing the most.

To interpret the difference in tree cover response of the models, it is helpful to be aware of some of the known climate and dynamic features of each model.
Firstly, CanESM5 particularly stands out as having a net loss of tree area by 2100 (Fig. \ref{fig:land_use}a), however, this is due to CanESM5 lacking an explicit representation of shrubs and rangelands, which have been allocated as forest.
CanESM5, MPI-ESM and UKESM1-0-LL also feature substantial Amazonian die-back in both scenarios, typically driven by localized drying.
Secondly, MPI-ESM1-2-LR has large amounts of tree cover increase in semi-arid regions in Africa and Australia.
Thirdly, the UKESM1-0-LL \textit{esm-ssp585-ssp126Lu} scenario is known to have enhanced CO$_2$ fertilization compared to other models and warming in the mid- to high-latitudes (resulting in increased tree cover fractions in Fig. S\ref{fig:tree_area_maps_ssp126Lu}) and decreased tree cover fraction in tropical South America and southeast Asia, driven by a combination of land use change and regional drying trends.

ACCESS-ESM1-5 is an example that closely follows the spatial distribution of the LUH2 land-use forcing.
By 2100, ACCESS-ESM1-5 has a forest expansion of 1.59 million km$^2$ and agricultural abandonment of 1.11 million km$^2$.
By mid-century, crops reach a minimum of 2.74 million km$^2$ less than in 2015, before rising again in the latter half of the century (Fig. S\ref{fig:ACCESS_land_use}).
Forestation is dominated by growth of evergreen broad leaf forests, followed by evergreen needle leaf forests, and deciduous broad leaf forests.
Deciduous needle leaf forests only make up a small fraction of forests and do not show any expansion.

\begin{figure}[H]
    \centering
    \begin{subfigure}[b]{0.8\linewidth}
        \includegraphics[width=\linewidth]{../plots/tree_area_ssp126_all_models_global_sum.png}
    \end{subfigure}
    \begin{subfigure}[b]{0.8\linewidth}
        \includegraphics[width=\linewidth]{../plots/tree_frac_map_2015-2100_diff.png}
    \end{subfigure}
    \caption{Global mean tree cover area changes between 2015 to 2100 in the forestation scenario (solid) and reference simulation (dashed) for each model. 2100 maps of tree cover fraction difference between the simulations for each model. MIROC-ES2L data for tree fraction are not available.}
    \label{fig:land_use_map}
\end{figure}

\begin{figure}[H]
    \centering
    \begin{subfigure}[b]{0.45\linewidth}
        \includegraphics[width=\linewidth]{../plots/cLand_model_intercomparison_diff.png}
    \end{subfigure}
    \begin{subfigure}[b]{0.45\linewidth}
        \includegraphics[width=\linewidth]{../plots/cVeg_model_intercomparison_diff.png}
    \end{subfigure}
    \begin{subfigure}[b]{0.45\linewidth}
        \includegraphics[width=\linewidth]{../plots/cSoil+cLitter_model_intercomparison_diff.png}
    \end{subfigure}
    \caption{Differences for cLand, cVeg and cLitter+cSoil between the \textit{esm-ssp585-ssp126Lu} and the \textit{esm-ssp585} scenario for 6 CMIP6 models. ACCESS-ESM1-5 is plotted as the ensemble mean and the blue shading indicates the ACCESS-ESM1-5 ensemble range.}
    \label{fig:models_cpools}
\end{figure}

The diversity of increased tree cover fraction among the models corresponds to a diversity of carbon draw-down potentials into the terrestrial system.
Figure \ref{fig:models_cpools} shows the change in the model terrestrial carbon pools due to forestation.
The increase in total land carbon tends to diminish towards the end of the century as new forest areas reach maturity.
The models show a total CO$_2$ removal by the land surface of between 10--60 Pg C by 2100.
The ACCESS-ESM1-5 ensemble spread indicates that the internal climate variability can constitute a considerable portion of this range (between 10--40 Pg C).

For vegetation carbon, the models either maintain a draw-down potential of $\sim$20--50 Pg C by 2100 (ACCESS-ESM1-5, UKESM1-0-LL, MPI-ESM1-2-LR and CESM2), or the vegetation gains in the middle of the century have been lost by 2100 (CanESM5 and MIROC-ES2L).
ACCESS-ESM1-5 and UKESM1-0-LL are approximately in the middle of the model spread.
These models share the same atmospheric model component and therefore share many of the same climate physics, however UKESM has a greater transient climate response to forestation and atmospheric CO$_2$ changes.

Soil carbon shows varied responses to forestation, but most models show carbon accumulates in litter and soil pools and that remain long-term sinks over the 21st century.
For example, for CanESM5, even though there is a net loss of tree cover by 2100, much of the land carbon is stored in soil and litter.
However, ACCESS-ESM1-5 and UKESM1-0-LL show decreases in soil carbon in response to forestation.
For ACCESS-ESM1-5, this is likely due to difference in PFT specific parameters for litter lignin ratio and litter and soil carbon turnover rates between forests, crops and grasses, with the former having slower turnover from litter to soil.
This results in carbon accumulating in the litter pools and the soil cabon pools decay to a new lower equlibrium.
In the UKESM1-0-LL, however, severe deforestation of tropical PFTs in the early part of SSP585 compared to SSP126 results in a large negative difference litter and soil carbon mid-century.
Much of the deforested wood is transferred to wood products, with less harvested carbon being transferred to soil in the esm-ssp585-ssp126Lu scenario.
The soil mid-century negative difference in soil carbon is somewhat recovered
by the end of the century due to an increase in forestation in the SSP585 scenario.

\begin{figure}
    \centering
    \includegraphics[width=\linewidth]{../plots/models_co2_diff.png}
    \caption{Difference in atmospheric CO2 concentration between \textit{esm-ssp585} and \textit{esm-ssp585-ssp126Lu}}
    \label{fig:models_CO2}
\end{figure}

The effect of an increased land surface sink is a corresponding decrease in atmospheric CO$_2$ concentrations as demonstrated by (Fig. \ref{fig:models_CO2}).
The multi-model range is -5 to -22 ppm, and with concentrations projected to increase from 400 ppm to 1088 ppm under SSP5-8.5 (REMIND-MAGPIE in Fig. S\ref{fig:models_co2_absolute}).
This change represents 0.7–3\% of the reduction required to return the CO$_2$ concentration at 2100 to that of the level in 2010.
The largest change in concentration is $\sim$22 ppm from MPI-ESM1-2-LR, which is still much lower than the 85 ppm decrease in the scenario used by \cite{sonntag_reforestation_2016}.
The ACCESS-ESM1-5 ensemble range demonstrates that atmospheric CO$_2$ is strongly sensitive to internal climate variability, encompassing 60\% of the multi model range.

\subsubsection{Climate response}

Fig. \ref{fig:models_tas_trends} shows that trends in global mean surface air temperature over the course of the century is not significant in all models.
Furthermore, there is some disagreement in the sign of the change in global temperature among models and ensemble members, with CanESM5 and CESM2 showing non-significant positive trends.
The effect that internal climate variability can have on the trends is demonstrated by the ACCESS-ESM1-5 ensemble.
While the ensemble mean trend showed no significant change, three members showed a significant positive trend.
The CO$_2$ concentration of these three members is not consistently greater than the other ensemble members throughout the experiment, which indicates that the significance of the temperature decrease in these members is mostly driven by internal climate variability.

The temporal variance of temperature also shows unique behavior among the models.
For example, MIROC-ES2L features large decadal-scale oscillations in global mean temperature driven by large El Niño–Southern Oscillation amplitude that results in similar variability in global temperature (\cite{hajima_development_2020}).
This occurs in both the forestation scenario and the reference scenario (Fig. S\ref{fig:models_absolute}), which causes large oscillations in the difference as they drift out of phase in the latter half of the century.

Similar to global mean surface air temperature, the response of global mean precipitation to forestation is also unclear from the models (Fig. S\ref{fig:models_pr_trends}), with all models showing no significant trends in global precipitation rate.

\begin{figure}[H]
    \centering
    \includegraphics[width=\linewidth]{../plots/tas_trends.png}
    \caption{Surface air temperature trends in the difference between the forestation scenario and the reference simulation. Solid lines indicate trends that are statistically significant at the 5\% level, and dotted lines are not significant trends. (+) and (-) denote the sign of the trend.}
    \label{fig:models_tas_trends}
\end{figure}

\subsection{Regional land carbon and climate responses in ACCESS-ESM1.5}

\subsubsection{Overview of ACCESS-ESM1.5 response to forestation}

Since ACCESS-ESM1-5 has 10 ensemble members available, and the regional distribution of new forest growth varies greatly among the models, the regional analysis will focus only on ACCESS-ESM1-5.
The single model ensemble allows us to examine the impact of forestation on the full distribution of regional surface temperatures and carbon uptake.
The carbon cycle in ACCESS-ESM1-5 in the forestation scenario reflects the temporal pattern of forestation in the forcing data well (Fig. \ref{fig:global_carbon_budget}a), with the land surface acting as a sink in the first half of the century when most of the forestation occurs and becoming a weak source towards the end of the century.
The ocean sink strengthens as the partial pressure of CO$_2$ on the ocean surface increases throughout the century from increasing atmospheric CO$_2$ concentrations.
However, the lower atmospheric CO$_2$ concentrations relative to the reference simulation results in the ocean absorbing cumulatively $\sim$1.3±0.5 Pg C less by 2100 (Fig. \ref{fig:global_carbon_budget}b).
Globally, cLand increases by 1.3±0.3\% of the cLand in the reference simulation, with 3.3±0.4 increase in cVeg and 0.5±3 decrease in cSoil.
ACCESS-ESM1-5 is also the only model to include phosphorus nutrient limitation, and therefore the ACCESS-ESM1-5 cVeg pools contrasts to other models, reaching a stable limit by 2100 while other model's cVeg are still increasing by 2100 (Fig. S\ref{fig:models_absolute}b).
The climate in both the forestation and reference simulations are similar, with a warming of $\sim$4 \textcelsius{} by 2100 and precipitation increases by 0.216 kg m$^{-2}$ day$^{-1}$.

\begin{figure}[H]
    \centering
    \begin{subfigure}[b]{0.4\linewidth}
        \includegraphics[width=\linewidth]{../plots/esm-ssp585-ssp126Lu_budget.png}
    \end{subfigure}
    \begin{subfigure}[b]{0.4\linewidth}
        \includegraphics[width=\linewidth]{../plots/fgco2_ocean_carbon_aff_ssp585_cumulative.png}
    \end{subfigure}
    \caption{Carbon budget of global fluxes of fossil fuel emissions and the natural sinks for the land (net-land-atmosphere exchange as the sum of the natural terrestrial sink and land-use change fluxes), ocean and atmospheric accumulation. The cumulative ocean carbon difference between the forestation and reference simulation.}
    \label{fig:global_carbon_budget}
\end{figure}

%\begin{table}[H]
%    \centering
%    \begin{tabular}{@{}llll@{}}
%    \hline
%        & $\Delta$ Carbon (Pg) & \% present day & \% of 2080-2100 ssp585 \\ \hline
%cLand   & 22.0±5.3    & 1.3±0.3        & 1.3±0.3                \\
%cVeg    & 24.8±2.7    & 3.7±0.4        & 3.3±0.4                \\
%cLitter & 0.4±0.3     & 0.8±0.6        & 0.9±0.7                \\
%cSoil   & -4.3±2.3    & -0.5±0.3       & -0.5±0.3               \\ \hline
%    \end{tabular}
%    \caption{C pools expressed as a percentage of the present day carbon content or  as a percentage of the end of \textit{esm-ssp585}. This information can mostly be seen in Fig. 4 and is only presented here to show the options on how to present the data. Will likely be removed in favor of quoting in text.}
%    \label{tab:cpools_table}
%\end{table}

\subsubsection{Regional changes in vegetation and climate extremes}

There are increases in land carbon for all regions and ensemble members except Amazonia, where some ensemble members show a small decrease in cVeg by 2100, due to internal climate variations.
The region with the largest change in land carbon content is Central Africa (Fig. \ref{fig:accesss_regional_cland}e), however this difference is due to avoided deforestation that occurs in the reference simulation, rather than due to new forest growth in the forestation experiment (Fig. S\ref{fig:ACCESS_land_cover}b).
This highlights the importance of including avoided deforestation in future long-term national climate strategies, not just to avoid related CO2 emissions from the burning and decay of biomass and soil carbon, but also since a considerable portion of land-use emissions comes from the loss of additional sink capacity from deforestation \parencite{gitz_amplifying_2003, pongratz_terminology_2014, obermeier_modelled_2021}.
The increased CO$_2$ draw-down from the combined effect of forestation and CO$_2$ fertilization are partially offset by the increase in soil respiration (a feature that only occurs in one other model, Fig. \ref{fig:access_cpools}), particularly in Australia where there is no increase in forest cover in the SSP1-2.6 scenario.

\begin{figure}[H]
    \centering
    \begin{subfigure}[b]{\linewidth}
        \includegraphics[width=\linewidth]{../plots/cLand_all_reagions.png}
    \end{subfigure}

%    \begin{subfigure}[b]{0.4\linewidth}
%        \includegraphics[width=\linewidth]{../plots/cLand_easternnorthamerica_diff.png}
%    \end{subfigure}
%    \begin{subfigure}[b]{0.4\linewidth}
%        \includegraphics[width=\linewidth]{../plots/cLand_eastasia_diff.png}
%    \end{subfigure}
%    \begin{subfigure}[b]{0.4\linewidth}
%        \includegraphics[width=\linewidth]{../plots/cLand_borealnorthamerica_diff.png}
%    \end{subfigure}
%    \begin{subfigure}[b]{0.4\linewidth}
%        \includegraphics[width=\linewidth]{../plots/cLand_centralafrica_diff.png}
%    \end{subfigure}
    \caption{Differences between the forestation experiment and the reference simulation for total land carbon content for each region in Fig. \ref{fig:land_use}.}
    \label{fig:accesss_regional_cland}
\end{figure}

The relationship of surface air temperature and changes in total tree cover fraction varies substantially by region.
For example, in Fig. \ref{fig:map_tas_tree_correlation}, the correlation of temperature and tree cover fraction is positive in the tropical regions of Central Africa, South America, the Maritime Continent, and East Asia.
Hence, increased tree cover fraction increases surface air temperature and the effect of decreased surface albedo dominates.
In contrast, Central Africa, which surrounds the avoided deforestation in the reference simulations, shows the opposite effect, whereby increased tree cover negatively correlates with air temperature and hence the cooling effect of evapotranspiration dominates.
Furthermore, in the sub-tropical and boreal regions of eastern North America, the correlation is negative, indicating that as forest cover increases, temperature decreases.

\begin{figure}[H]
    \centering
    \begin{subfigure}[b]{0.7\linewidth}
        \includegraphics[width=\linewidth]{../plots/correlation_tree_tas_ens_mean_first.png}
    \end{subfigure}
%    \begin{subfigure}[b]{0.5\linewidth}
%        \includegraphics[width=\linewidth]{../plots/correlation_tree_tas_significance.png}
%    \end{subfigure}
    \caption{Correlation of ensemble mean 2 m surface air temperature (\textit{esm-ssp585-ssp126Lu} - \textit{esm-ssp585}) with tree fraction in ACCESS-ESM1-5. Only statistically significant correlations at the 5\% level are shown.}
    \label{fig:map_tas_tree_correlation}
\end{figure}

The biogeophysical impacts of growing trees cause localized warming at the extreme ends of the temperature distribution, especially in the higher latitudes where trees tend to be slow growing and the ground having a high albedo from snow cover.
This warming may offset the biogeochemical climate mitigation potential of forestation.
For specific grid-points with large changes from grass to tree biomes, the distribution of summer daily maximum surface air temperature for both the forestation and reference simulations are shown in Figure \ref{fig:tasmax_distribution}.
The Amazon grid-point features changes in mostly C4 grass to evergreen broad leaf forest, representing an increase in tree cover fraction of 60\%.
This corresponds to a statistically significant change in the distribution, particularly for temperatures greater than 50 \textcelsius{} (Fig. \ref{fig:tasmax_distribution}b).

The increase in the high end of the distribution in response to forestation is not consistent for all regions.
For example, the large increase in forest cover for the grid point in Asia corresponds to a decrease in temperatures greater than 23 \textcelsius{}, while temperatures 17--23 \textcelsius{} increase (Fig. \ref{fig:tasmax_distribution}d), as the distribution gets narrower with forestation.
The changes in daily maximum temperature at the lower end of the distribution in response to forestation are much more regionally consistent, showing a decrease in cooler than average days.

Some regions show decreases in surface air temperature in response to increasing tree cover.
Of particular note is Eastern North America which features a large transition from C3 crops to deciduous broadleaf (Fig. \ref{fig:tasmax_distribution} e).
In ACCESS-ESM1-5, deciduous broad leaf forests have the highest reflectance of all the PFTs.
The resulting distribution shows decreases in the number of warm and cool days in the forestation experiment (Fig. \ref{fig:tasmax_distribution} f).

\begin{figure}[H]
    \centering
    \begin{subfigure}[b]{0.4\linewidth}
        \includegraphics[width=\linewidth]{../plots/pfts_amazongridpoint.png}
    \end{subfigure}
    \begin{subfigure}[b]{0.4\linewidth}
        \includegraphics[width=\linewidth]{../plots/tasmax_histogram_amazongridpoint.png}
    \end{subfigure}
%    \begin{subfigure}[b]{0.4\linewidth}
%        \includegraphics[width=\linewidth]{../plots/pfts_easternnorthamericagridpoint.png}
%    \end{subfigure}
%    \begin{subfigure}[b]{0.4\linewidth}
%        \includegraphics[width=\linewidth]{../plots/tasmax_histogram_easternnorthamerica.png}
%    \end{subfigure}
    \begin{subfigure}[b]{0.4\linewidth}
        \includegraphics[width=\linewidth]{../plots/pfts_asiagridopint.png}
    \end{subfigure}
    \begin{subfigure}[b]{0.4\linewidth}
        \includegraphics[width=\linewidth]{../plots/tasmax_histogram_asiagridopint.png}
    \end{subfigure}
    \begin{subfigure}[b]{0.4\linewidth}
        \includegraphics[width=\linewidth]{../plots/pfts_northamericagridpoint.png}
    \end{subfigure}
    \begin{subfigure}[b]{0.4\linewidth}
        \includegraphics[width=\linewidth]{../plots/tasmax_histogram_northamericagridpoint.png}
    \end{subfigure}
    \caption{Left: Changes in plant functional types for select grid points in \textit{esm-ssp585-ssp126Lu} in ACCESS-ESM1-5. Right: Distributions of summer time (June--August or December--February) maximum daily temperature for the last 20 years of \textit{esm-ssp585-ssp126Lu} (green) and \textit{esm-ssp585} (yellow).}
    \label{fig:tasmax_distribution}
\end{figure}

\section{Concluding remarks}

We conduct a multi-model intercomparison of a scenario for forestation as a means of carbon dioxide removal.
This forestation scenario features high fossil fuel emissions, a much warmer climate and moderate forestation and agricultural abandonment.
Models show a diverse interpretation of the spatial patterns of forestation, and as a result show a large range of outcomes for long-term carbon storage in forests.
Four models show a stable but limited carbon sink by 2100, while two models show that the mitigation gains from forestation in the middle of the century will be mostly lost by 2100.

The change in atmospheric CO$_2$ concentrations from forestation only accounts for 0.7--3\% of the reduction required to return the SSP5-8.5 concentrations at 2100 to those at 2010.
As a result, the models show that forestation causes only a small impact on global climate under a high emissions and high global warming scenario.
The forestation also causes a shift in the global carbon balance, whereby increased uptake of carbon on the land of $\sim$25 Pg C by 2100 results in a decrease in the uptake of carbon by the ocean.
There are some increases in local-scale temperatures in locations where forestation occurs, while other regions show cooling.

The scenario used in this study is specific to a world of extreme CO$_2$ emissions, and does not consider the case where significant reduction in fossil fuel emissions occur.
It is therefore still unclear how much more or less carbon would be sequestered by the terrestrial ecosystem under a cooler climate that would occur in conjunction with the expected emissions reduction efforts in the future.
Therefore, future studies should aim to explore the effects of forestation for climate states under different target warming levels that are consistent with the Paris Agreement  (for example \cite{king_studying_2021}).

While it is clear that most models show that growing new forests under a very high emissions scenario has limited climate mitigation potential, this does not mean that forestation does not have a role in climate mitigation.
In addition to climate benefits, forestation and forest management provides a broad range of co-benefits such as increased habitat, biodiversity and soil protection, and many of these features are not yet simulated in Earth system models, nor is the additional benefit of these ecosystem services accounted for in climate policies.
For forestation to be an efficient long-term CO2 removal strategy, it must also exist in conjunction with other strategies.
By first regrowing forests for the purpose of CO$_2$ removal, forestation increases the natural land-based carbon and enables further development and supply of feedstock for human activity, including for climate mitigation.
Forests that are sustainably harvested and regrown to remove CO$_2$ act as long-term carbon sinks both in soils and in harvested wood products.
Vegetation carbon may be lost in individual natural disturbance events such as fires, but the historically removed carbon remains locked.
None of the models in this study (and very few in general) fully implement nature- and technology-based removal strategies, and therefore do not account for forests to be leveraged in bio-energy with carbon capture and storage.
Since forestation and bio-energy usage is a key assumption of many low-emissions SSP scenarios to replace fossil fuels, implementing them in the Earth system modeling context is important for future research.

\section{Acknowledgments}
We would like to thank the support provided by the National Environmental Science Program - Climate Systems Hub.

\printbibliography

\section{Supplementary material}
\setcounter{figure}{0}

\subsection{Tree cover fractions}

\begin{figure}[H]
    \centering
    \includegraphics[width=\linewidth]{../plots/Figure_S1.png}
    \caption{LUH2 primary and secondary forest cover fractions difference between SSP1-2.6 and SSP5-8.5.}
    \label{fig:LUH2DIFF}
\end{figure}

\begin{figure}[H]
    \centering
    \includegraphics[width=\linewidth]{../plots/tree_fraction_ssp126_all_models_globalmean.png}
    \caption{Tree cover fractions output for each model. The solid lines are the \textit{esm-ssp585-ssp126Lu} simulation and the dashed lines are the \textit{esm-ssp585} simulation.}
    \label{fig:tree_fractions_models}
\end{figure}

\begin{figure}[H]
    \centering
    \begin{subfigure}[b]{0.4\linewidth}
        \includegraphics[width=\linewidth]{../plots/tree_frac_map_2015-2100_for_delta.png}
    \end{subfigure}
    \begin{subfigure}[b]{0.4\linewidth}
        \includegraphics[width=\linewidth]{../plots/tree_frac_map_2015-2100_ssp585_delta.png}
    \end{subfigure}
%    \begin{subfigure}[b]{0.4\linewidth}
%        \includegraphics[width=\linewidth]{../plots/tree_frac_map_2015-2100_diff.png}
%    \end{subfigure}
    \caption{Tree cover fractions map for all models expressed as a temporal anomaly. Left: Forestation esperiment. Right: SSP585.}
    \label{fig:tree_area_maps_ssp126Lu}
\end{figure}

\begin{figure}[H]
    \centering
    \begin{subfigure}[b]{0.4\linewidth}
        \includegraphics[width=\linewidth]{../plots/treeFrac_esm-ssp585_anomaly.png}
    \end{subfigure}
    \begin{subfigure}[b]{0.4\linewidth}
        \includegraphics[width=\linewidth]{../plots/grassFrac_esm-ssp585_anomaly.png}
    \end{subfigure}
    \begin{subfigure}[b]{0.4\linewidth}
        \includegraphics[width=\linewidth]{../plots/cropFrac_esm-ssp585_anomaly.png}
    \end{subfigure}
    \caption{Temporal anomaly in area of trees, grass and crop between 2015 and 2100 for \textit{esm-ssp585} in ACCESS-ESM1.5.
    % This should probably be the LUH2 grass and crop plots for both scenarios.
    }
    \label{fig:ACCESS_land_cover}
\end{figure}

\begin{figure}[H]
    \centering
    \begin{subfigure}[b]{0.45\linewidth}
        \includegraphics[width=\linewidth]{../plots/land_areas.png}
    \end{subfigure}
    \begin{subfigure}[b]{0.45\linewidth}
        \includegraphics[width=\linewidth]{../plots/land_areas_diff.png}
    \end{subfigure}
    \begin{subfigure}[b]{0.45\linewidth}
        \includegraphics[width=\linewidth]{../plots/CABLE_forests.png}
    \end{subfigure}
    \caption{Land use areas in ACCESS-ESM1.5. Left: ssp585 (dashed) and forestation scenario (solid). Right: difference between ssp585 and forestation scenario.}
    \label{fig:ACCESS_land_use}
\end{figure}

\subsection{CO2}

\begin{figure}[H]
    \centering
    \includegraphics[width=\linewidth]{../plots/models_co2.png}
    \caption{Absolute CO2}
    \label{fig:models_co2_absolute}
\end{figure}

\subsection{Climate}

\begin{figure}[H]
    \centering
    \includegraphics[width=\linewidth]{../plots/pr_trends.png}
    \caption{Surface global precipitation rate trends in the difference between the forestation scenario and the reference simulaiton. Solid lines indicate trends that are statistically significant at the 5\% level, and dotted lines are not significant trends. `+' and `-' denote the sign of the trend.}
    \label{fig:models_pr_trends}
\end{figure}

\subsection{Absolute values}

\begin{figure}[H]
    \centering
    \begin{subfigure}[b]{0.45\linewidth}
        \includegraphics[width=\linewidth]{../plots/cLand_model_intercomparison_esm-ssp585-ssp126Lu.png}
    \end{subfigure}
    \begin{subfigure}[b]{0.45\linewidth}
        \includegraphics[width=\linewidth]{../plots/cVeg_model_intercomparison_esm-ssp585-ssp126Lu.png}
    \end{subfigure}
    \begin{subfigure}[b]{0.45\linewidth}
        \includegraphics[width=\linewidth]{../plots/cLitter_model_intercomparison_esm-ssp585-ssp126Lu.png}
    \end{subfigure}
    \begin{subfigure}[b]{0.45\linewidth}
        \includegraphics[width=\linewidth]{../plots/cSoil_model_intercomparison_esm-ssp585-ssp126Lu.png}
    \end{subfigure}
    \begin{subfigure}[b]{0.45\linewidth}
        \includegraphics[width=\linewidth]{../plots/tas_model_intercomparison_esm-ssp585-ssp126Lu.png}
    \end{subfigure}
    \begin{subfigure}[b]{0.45\linewidth}
        \includegraphics[width=\linewidth]{../plots/pr_model_intercomparison_esm-ssp585-ssp126Lu.png}
    \end{subfigure}
    \caption{Absolute values of the forestation scenario's global mean carbon pools, 2 m surface air temperature and precipitation for each model. The solid line for ACCESS ESM-1-5 is the ensemble mean and the shading indicates the ensemble range.}
    \label{fig:models_absolute}
\end{figure}

%\begin{figure}[H]
%    \centering
%    \begin{subfigure}[b]{0.45\linewidth}
%        \includegraphics[width=\linewidth]{../plots/cLand_model_intercomparison_esm-ssp585.png}
%    \end{subfigure}
%    \begin{subfigure}[b]{0.45\linewidth}
%        \includegraphics[width=\linewidth]{../plots/cVeg_model_intercomparison_esm-ssp585.png}
%    \end{subfigure}
%    \begin{subfigure}[b]{0.45\linewidth}
%        \includegraphics[width=\linewidth]{../plots/cLitter_model_intercomparison_esm-ssp585.png}
%    \end{subfigure}
%    \begin{subfigure}[b]{0.45\linewidth}
%        \includegraphics[width=\linewidth]{../plots/cSoil_model_intercomparison_esm-ssp585.png}
%    \end{subfigure}
%    \caption{Absolute values of the SSP5-8.5 scenario's global mean carbon pools, 2m surface air temperature and precipitation for each model. The solid line for ACCESS ESM-1-5 is the ensemble mean and the shading indicates the ensemble range.}
%    \label{fig:models_absolute2}
%\end{figure}

\subsection{ACCESS-ESM1-5 global pools}

\begin{figure}[H]
    \centering
    \begin{subfigure}[b]{0.4\linewidth}
        \includegraphics[width=\linewidth]{../plots/cVeg_ACCESS-ESM1.5_esm-ssp585-ssp126Lu_ensembles_anomalies.png}
    \end{subfigure}
    \begin{subfigure}[b]{0.4\linewidth}
        \includegraphics[width=\linewidth]{../plots/cLitter_ACCESS-ESM1.5_esm-ssp585-ssp126Lu_ensembles_anomalies.png}
    \end{subfigure}
    \begin{subfigure}[b]{0.4\linewidth}
        \includegraphics[width=\linewidth]{../plots/cSoil_ACCESS-ESM1.5_esm-ssp585-ssp126Lu_ensembles_anomalies.png}
    \end{subfigure}
\begin{subfigure}[b]{0.4\linewidth}
        \includegraphics[width=\linewidth]{../plots/cLand_ACCESS-ESM1.5_esm-ssp585-ssp126Lu_ensembles_anomalies.png}
    \end{subfigure}
    \caption{ACCESS-ESM1-5 carbon pools. \textit{esm-ssp585-ssp126Lu} relative to the 2005--2025 period.}
    \label{fig:access_cpools}
\end{figure}

%\begin{figure}[H]
%    \centering
%    \begin{subfigure}[b]{0.4\linewidth}
%        \includegraphics[width=\linewidth]{../plots/gpp_ACCESS-ESM1.5_esm-ssp585-ssp126Lu_ensembles_anomalies.png}
%    \end{subfigure}
%    \begin{subfigure}[b]{0.4\linewidth}
%        \includegraphics[width=\linewidth]{../plots/gpp_ACCESS-ESM1.5_esm-ssp585-ssp126Lu_ensembles_diff.png}
%    \end{subfigure}
%    \begin{subfigure}[b]{0.4\linewidth}
%        \includegraphics[width=\linewidth]{../plots/npp_ACCESS-ESM1.5_esm-ssp585-ssp126Lu_ensembles_anomalies.png}
%    \end{subfigure}
%    \begin{subfigure}[b]{0.4\linewidth}
%        \includegraphics[width=\linewidth]{../plots/npp_ACCESS-ESM1.5_esm-ssp585-ssp126Lu_ensembles_diff.png}
%    \end{subfigure}
%    \begin{subfigure}[b]{0.4\linewidth}
%        \includegraphics[width=\linewidth]{../plots/ra_ACCESS-ESM1.5_esm-ssp585-ssp126Lu_ensembles_anomalies.png}
%    \end{subfigure}
%    \begin{subfigure}[b]{0.4\linewidth}
%        \includegraphics[width=\linewidth]{../plots/ra_ACCESS-ESM1.5_esm-ssp585-ssp126Lu_ensembles_diff.png}
%    \end{subfigure}
%    \begin{subfigure}[b]{0.4\linewidth}
%        \includegraphics[width=\linewidth]{../plots/rh_ACCESS-ESM1.5_esm-ssp585-ssp126Lu_ensembles_anomalies.png}
%    \end{subfigure}
%    \begin{subfigure}[b]{0.4\linewidth}
%        \includegraphics[width=\linewidth]{../plots/rh_ACCESS-ESM1.5_esm-ssp585-ssp126Lu_ensembles_diff.png}
%    \end{subfigure}
%    \begin{subfigure}[b]{0.4\linewidth}
%        \includegraphics[width=\linewidth]{../plots/nbp_ACCESS-ESM1.5_esm-ssp585-ssp126Lu_ensembles_anomalies.png}
%    \end{subfigure}
%    \begin{subfigure}[b]{0.4\linewidth}
%        \includegraphics[width=\linewidth]{../plots/nbp_ACCESS-ESM1.5_esm-ssp585-ssp126Lu_ensembles_diff.png}
%    \end{subfigure}
%    \caption{Carbon fluxes in \textit{esm-ssp585-ssp126Lu}.  Left: relative to 2005--2025. Right: relative to \textit{esm-ssp585}.}
%    \label{fig:access_cflux}
%\end{figure}

%\begin{figure}[H]
%    \centering
%    \begin{subfigure}[b]{0.45\linewidth}
%        \includegraphics[width=\linewidth]{../plots/tas_ACCESS-ESM1.5_esm-ssp585_ensembles.png}
%    \end{subfigure}
%    \begin{subfigure}[b]{0.45\linewidth}
%        \includegraphics[width=\linewidth]{../plots/tas_ACCESS-ESM1.5_esm-ssp585_ensembles_diff.png}
%    \end{subfigure}
%    \begin{subfigure}[b]{0.45\linewidth}
%        \includegraphics[width=\linewidth]{../plots/pr_ACCESS-ESM1.5_esm-ssp585_ensembles.png}
%    \end{subfigure}
%    \begin{subfigure}[b]{0.45\linewidth}
%        \includegraphics[width=\linewidth]{../plots/pr_ACCESS-ESM1.5_esm-ssp585_ensembles_diff.png}
%    \end{subfigure}
%    \caption{Climate in \textit{esm-ssp585-ssp126Lu}. Left: absolute values from the forestation simulation (\textit{esm-ssp585-ssp126Lu}). Right: difference between the \textit{esm-ssp585-ssp126Lu} simulation and \textit{esm-ssp585}}
%    \label{fig:climate}
%\end{figure}

%\subsection{Regional analysis}
%
%\begin{figure}[H]
%    \centering
%    \begin{subfigure}[b]{0.4\linewidth}
%        \includegraphics[width=\linewidth]{../plots/tas_amazonia_diff.png}
%    \end{subfigure}
%    \begin{subfigure}[b]{0.4\linewidth}
%        \includegraphics[width=\linewidth]{../plots/tas_easternnorthamerica_diff.png}
%    \end{subfigure}
%    \begin{subfigure}[b]{0.4\linewidth}
%        \includegraphics[width=\linewidth]{../plots/tas_eastasia_diff.png}
%    \end{subfigure}
%    \begin{subfigure}[b]{0.4\linewidth}
%        \includegraphics[width=\linewidth]{../plots/tas_borealnorthamerica_diff.png}
%    \end{subfigure}
%    \begin{subfigure}[b]{0.4\linewidth}
%        \includegraphics[width=\linewidth]{../plots/tas_centralafrica_diff.png}
%    \end{subfigure}
%    \caption{Mean 2m surface air temperature for each region in ACCESS-ESM1-5.}
%    \label{fig:ACCESS_tas_regions}
%\end{figure}

\end{document}
