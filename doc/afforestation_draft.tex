% Preamble.
\documentclass[]{article}
\usepackage[a4paper, total={170mm,257mm}, left=20mm, top=20mm,]{geometry}
\usepackage{svg} % Use SVGs only for final draft.
\usepackage{float} % Allows the [H] option meaning "place figure here and only here"
%\usepackage[section]{placeins} % Defines /FloatBarrier so that figures can't move.
\usepackage{subcaption} % Allows subfigures and captions
\usepackage{lineno} % Enable line numbers.
\usepackage{textcomp}
\usepackage[style=numeric]{biblatex}
\addbibresource{afforestation_references.bib}
\linenumbers
\title{Afforestation in ACCESS-ESM1.5 \& maybe other models too... dunno yet}
\author{Tammas Loughran et al.}

\begin{document}

\maketitle

\section{Introduction}

A considerable amount of anthropogenic carbon emissions comes from agricultural activities.
In order to effectively limit global warming to below 2°C (or even 3°C), there must be a shift in the reliance of global diets away from agricultural practices that emit greenhouse gasses into the atmosphere (via e.g. forest clearing or ruminating herbivores).
This shift typically implies widespread abandonment of agricultural lands and an increase in natural vegetation (known as reforestation and afforestation of lands were not previously occupied by forest).
This study is motivated by existing studies on afforestation by Sonntag et al. (2016) where the potential of afforestation is quantified using and Earth System Model, where a high CO2 emissions scenario is simulated (taken from RCP8.5) in combination with the "afforestation" land use from a low emissions scenario (taken from RCP4.5).
This previous study used the concentration driven CMIP5 version of the MPI-ESM and the CMIP5 representative concentration pathways. This study aims to quantify the CO2 removal potential of the afforestation strategy using updated models at both the global scale and for Australia specifically.
As part of the contributions to the CMIP5 LUMIP experiments, modelling groups were requested to do emissions driven simulations of a similar afforestation scenario.
This time using the Shared Socioeconomic Pathways' emissions from SSP5-8.5 and the land use of SSP1-2.6. ACCESS-ESM1.5 contributed 10 ensemble members to these experiments.

\begin{itemize}
    \item existing studies on MPI-ESM CMIP5 version using RCP8.5 emissions with RCP4.5 land use. \cite{sonntag_reforestation_2016} \cite{sonntag_quantifying_2018}
    \item esm-hist: emissions-driven simulations are initialized at 2015 using the esm-hist. \cite{eyring_overview_2016}
    \item esm-ssp585: High emissions scenario that assumes that development is driven by fossil fuels. \cite{oneill_scenario_2016}
    \item esm-ssp585-ssp126Lu: High emissions scenario from SSP5-8.5 but with the land use from SSP1-2.6. \cite{lawrence_land_2016}
    \item SSP1-2.6: Challenges to adaptation and mitigation are both low. Land use changes characterized by decrease in pasture and crop and increase in natural lands. \cite{van_vuuren_energy_2017}
\end{itemize}

\section{Methods}

\begin{figure}[H]
    \centering
    \begin{subfigure}[b]{0.45\linewidth}
        %\includesvg[inkscape=force,width=\linewidth]{../plots/land_areas.svg}
        \includegraphics[width=\linewidth]{../plots/land_areas.png}
    \end{subfigure}
    \begin{subfigure}[b]{0.45\linewidth}
        %\includesvg[inkscape=force,width=\linewidth]{../plots/land_areas_diff.svg}
        \includegraphics[width=\linewidth]{../plots/land_areas_diff.png}
    \end{subfigure}
    \caption{Land use areas in ACCESS-ESM1.5. Left: ssp585 (dashed) and afforestation scenario (solid). Right: difference between ssp585 and afforestation scenario.}
    \label{fig:land_use}
\end{figure}

By the end of the century:
\begin{itemize}
    \item Forrest expands by: 1.59 million km²
    \item Abandonment: -1.11 million km²
\end{itemize}
Mid-century:
\begin{itemize}
    \item Crop minimum: -2.74 million km²
\end{itemize}

\begin{figure}[H]
    \centering
    \begin{subfigure}[b]{0.45\linewidth}
        %\includesvg[inkscape=force,width=\linewidth]{../plots/land_areas.svg}
        \includegraphics[width=\linewidth]{../plots/treeFrac_anomaly.png}
    \end{subfigure}
    \begin{subfigure}[b]{0.45\linewidth}
        %\includesvg[inkscape=force,width=\linewidth]{../plots/land_areas_diff.svg}
        \includegraphics[width=\linewidth]{../plots/grassFrac_anomaly.png}
    \end{subfigure}
    \caption{ACCESS-ESM1.5 Land use area change from 215 to 2100 in the afforestation scenario. Left: forest. Right: Grass. In ACCESS-ESM1.5 LUH2 pasture fractions are allocated to C3 grass.}
    \label{fig:land_use_map}
\end{figure}

Afforestation is dominated by growth of evergreen broadleaf forests, followed by evergreen needle leaf forests, then deciduous broadleaf forests.
Deciduous needle leaf forests only make a small fraction of forrests and do not show any expansion.
There are only a few areas where there is defforestation in the esm-ssp585-ssp126Lu experiment.
These occur in deciduous broadleaf forests in eastern USA, China and western Russia (Figure \ref{fig:land_use_map}).

\section{Results}

\subsection{Climate}

\begin{itemize}
    \item Climate warms about 4 \textcelsius{} by 2100 and precip. increases by 0.25e-5 kg m-2 s-1 \footnote{Need to change the units of this to something more human readable}
    \item There isn't much difference between the afforestation scenario and ssp585.
\end{itemize}

\begin{figure}[H]
    \centering
    \begin{subfigure}[b]{0.45\linewidth}
        %\includesvg[inkscape=force,width=\linewidth]{../plots/tas_ACCESS-ESM1-5_esm-ssp585_ensembles.svg}
        \includegraphics[width=\linewidth]{../plots/tas_ACCESS-ESM1-5_esm-ssp585_ensembles.png}
    \end{subfigure}
    \begin{subfigure}[b]{0.45\linewidth}
        %\includesvg[inkscape=force,width=\linewidth]{../plots/tas_ACCESS-ESM1-5_esm-ssp585_ensembles_diff.svg}
        \includegraphics[width=\linewidth]{../plots/tas_ACCESS-ESM1-5_esm-ssp585_ensembles_diff.png}
    \end{subfigure}
    \begin{subfigure}[b]{0.45\linewidth}
        %\includesvg[inkscape=force,width=\linewidth]{../plots/pr_ACCESS-ESM1-5_esm-ssp585_ensembles.svg}
        \includegraphics[width=\linewidth]{../plots/pr_ACCESS-ESM1-5_esm-ssp585_ensembles.png}
    \end{subfigure}
    \begin{subfigure}[b]{0.45\linewidth}
        %\includesvg[inkscape=force,width=\linewidth]{../plots/pr_ACCESS-ESM1-5_esm-ssp585_ensembles_diff.svg}
        \includegraphics[width=\linewidth]{../plots/pr_ACCESS-ESM1-5_esm-ssp585_ensembles_diff.png}
    \end{subfigure}
    \caption{Climate in esm-ssp585-ssp126Lu. Left: from the afforestation simulation. Right: difference between esm-ssp585-ssp126Lu simulation and esm-ssp585}
    \label{fig:climate}
\end{figure}

\subsection{Carbon cycle}

\begin{itemize}
    \item There is an increase in vegetation carbon of about 30 Pg C by the end of the century.
    \item The increased co2 draw-down from the combined effect of Afforestation and co2 fertilization seems to be offset by the increase in respiration in the litter and soil by the end of the century.
    \item cSoil is lower than SSP585 by about 4 Pg carbon.
\end{itemize}

\begin{figure}[H]
    \centering
    \begin{subfigure}[b]{0.4\linewidth}
        %\includesvg[inkscape=force,width=\linewidth]{../plots/cLitter_ACCESS-ESM1-5_esm-ssp585-ssp126Lu_ensembles_anomalies.svg}
        \includegraphics[width=\linewidth]{../plots/cLitter_ACCESS-ESM1-5_esm-ssp585-ssp126Lu_ensembles_anomalies.png}
    \end{subfigure}
    \begin{subfigure}[b]{0.4\linewidth}
        %\includesvg[inkscape=force,width=\linewidth]{../plots/cLitter_ACCESS-ESM1-5_esm-ssp585-ssp126Lu_ensembles_diff.svg}
        \includegraphics[width=\linewidth]{../plots/cLitter_ACCESS-ESM1-5_esm-ssp585-ssp126Lu_ensembles_diff.png}
    \end{subfigure}
    \begin{subfigure}[b]{0.4\linewidth}
        %\includesvg[inkscape=force,width=\linewidth]{../plots/cSoil_ACCESS-ESM1-5_esm-ssp585-ssp126Lu_ensembles_anomalies.svg}
        \includegraphics[width=\linewidth]{../plots/cSoil_ACCESS-ESM1-5_esm-ssp585-ssp126Lu_ensembles_anomalies.png}
    \end{subfigure}
    \begin{subfigure}[b]{0.4\linewidth}
        %\includesvg[inkscape=force,width=\linewidth]{../plots/cSoil_ACCESS-ESM1-5_esm-ssp585-ssp126Lu_ensembles_diff.svg}
        \includegraphics[width=\linewidth]{../plots/cSoil_ACCESS-ESM1-5_esm-ssp585-ssp126Lu_ensembles_diff.png}
    \end{subfigure}
    \begin{subfigure}[b]{0.4\linewidth}
        %\includesvg[inkscape=force,width=\linewidth]{../plots/cVeg_ACCESS-ESM1-5_esm-ssp585-ssp126Lu_ensembles_anomalies.svg}
        \includegraphics[width=\linewidth]{../plots/cVeg_ACCESS-ESM1-5_esm-ssp585-ssp126Lu_ensembles_anomalies.png}
    \end{subfigure}
    \begin{subfigure}[b]{0.4\linewidth}
        %\includesvg[inkscape=force,width=\linewidth]{../plots/cVeg_ACCESS-ESM1-5_esm-ssp585-ssp126Lu_ensembles_diff.svg}
        \includegraphics[width=\linewidth]{../plots/cVeg_ACCESS-ESM1-5_esm-ssp585-ssp126Lu_ensembles_diff.png}
    \end{subfigure}
    \caption{Carbon pools in esm-ssp585-ssp126Lu. Left: relative to 2005–2025. Right: relative to ssp-585.}
    \label{fig:cpools}
\end{figure}

\begin{figure}[H]
    \centering
    \begin{subfigure}[b]{0.4\linewidth}
        %\includesvg[inkscape=force,width=\linewidth]{../plots/gpp_ACCESS-ESM1-5_esm-ssp585-ssp126Lu_ensembles_anomalies.svg}
        \includegraphics[width=\linewidth]{../plots/gpp_ACCESS-ESM1-5_esm-ssp585-ssp126Lu_ensembles_anomalies.png}
    \end{subfigure}
    \begin{subfigure}[b]{0.4\linewidth}
        %\includesvg[inkscape=force,width=\linewidth]{../plots/gpp_ACCESS-ESM1-5_esm-ssp585-ssp126Lu_ensembles_diff.svg}
        \includegraphics[width=\linewidth]{../plots/gpp_ACCESS-ESM1-5_esm-ssp585-ssp126Lu_ensembles_diff.png}
    \end{subfigure}
    \begin{subfigure}[b]{0.4\linewidth}
        %\includesvg[inkscape=force,width=\linewidth]{../plots/npp_ACCESS-ESM1-5_esm-ssp585-ssp126Lu_ensembles_anomalies.svg}
        \includegraphics[width=\linewidth]{../plots/npp_ACCESS-ESM1-5_esm-ssp585-ssp126Lu_ensembles_anomalies.png}
    \end{subfigure}
    \begin{subfigure}[b]{0.4\linewidth}
        %\includesvg[inkscape=force,width=\linewidth]{../plots/npp_ACCESS-ESM1-5_esm-ssp585-ssp126Lu_ensembles_diff.svg}
        \includegraphics[width=\linewidth]{../plots/npp_ACCESS-ESM1-5_esm-ssp585-ssp126Lu_ensembles_diff.png}
    \end{subfigure}
    \begin{subfigure}[b]{0.4\linewidth}
        %\includesvg[inkscape=force,width=\linewidth]{../plots/ra_ACCESS-ESM1-5_esm-ssp585-ssp126Lu_ensembles_anomalies.svg}
        \includegraphics[width=\linewidth]{../plots/ra_ACCESS-ESM1-5_esm-ssp585-ssp126Lu_ensembles_anomalies.png}
    \end{subfigure}
    \begin{subfigure}[b]{0.4\linewidth}
        %\includesvg[inkscape=force,width=\linewidth]{../plots/ra_ACCESS-ESM1-5_esm-ssp585-ssp126Lu_ensembles_diff.svg}
        \includegraphics[width=\linewidth]{../plots/ra_ACCESS-ESM1-5_esm-ssp585-ssp126Lu_ensembles_diff.png}
    \end{subfigure}
    \begin{subfigure}[b]{0.4\linewidth}
        %\includesvg[inkscape=force,width=\linewidth]{../plots/rh_ACCESS-ESM1-5_esm-ssp585-ssp126Lu_ensembles_anomalies.svg}
        \includegraphics[width=\linewidth]{../plots/rh_ACCESS-ESM1-5_esm-ssp585-ssp126Lu_ensembles_anomalies.png}
    \end{subfigure}
    \begin{subfigure}[b]{0.4\linewidth}
        %\includesvg[inkscape=force,width=\linewidth]{../plots/rh_ACCESS-ESM1-5_esm-ssp585-ssp126Lu_ensembles_diff.svg}
        \includegraphics[width=\linewidth]{../plots/rh_ACCESS-ESM1-5_esm-ssp585-ssp126Lu_ensembles_diff.png}
    \end{subfigure}
    \begin{subfigure}[b]{0.4\linewidth}
        %\includesvg[inkscape=force,width=\linewidth]{../plots/nbp_ACCESS-ESM1-5_esm-ssp585-ssp126Lu_ensembles_anomalies.svg}
        \includegraphics[width=\linewidth]{../plots/nbp_ACCESS-ESM1-5_esm-ssp585-ssp126Lu_ensembles_anomalies.png}
    \end{subfigure}
    \begin{subfigure}[b]{0.4\linewidth}
        %\includesvg[inkscape=force,width=\linewidth]{../plots/nbp_ACCESS-ESM1-5_esm-ssp585-ssp126Lu_ensembles_diff.svg}
        \includegraphics[width=\linewidth]{../plots/nbp_ACCESS-ESM1-5_esm-ssp585-ssp126Lu_ensembles_diff.png}
    \end{subfigure}
    \caption{Carbon fluxes in esm-ssp585-ssp126Lu.  Left: relative to 2005–2025. Right: relative to ssp-585.}
    \label{fig:cflux}
\end{figure}

\pagebreak

\subsection{Regional analysis: Australia?}

\begin{itemize}
    \item Australian cVeg increases by ~0.25 PgC in the afforestation scenario. But the difference from the reference simulation is 0.
\end{itemize}

\begin{figure}[H]
    \centering
    \begin{subfigure}[b]{0.4\linewidth}
        %\includesvg[width=\linewidth]{../plots/cVeg_aus_anom.svg}
        \includegraphics[width=\linewidth]{../plots/cVeg_aus_anom.png}
    \end{subfigure}
    \begin{subfigure}[b]{0.4\linewidth}
        %\includesvg[width=\linewidth]{../plots/cVeg_aus_diff.svg}
        \includegraphics[width=\linewidth]{../plots/cVeg_aus_diff.png}
    \end{subfigure}
    \begin{subfigure}[b]{0.4\linewidth}
        %\includesvg[width=\linewidth]{../plots/cLitter_aus_anom.svg}
        \includegraphics[width=\linewidth]{../plots/cLitter_aus_anom.png}
    \end{subfigure}
    \begin{subfigure}[b]{0.4\linewidth}
        %\includesvg[width=\linewidth]{../plots/cLitter_aus_diff.svg}
        \includegraphics[width=\linewidth]{../plots/cLitter_aus_diff.png}
    \end{subfigure}
    \begin{subfigure}[b]{0.4\linewidth}
        %\includesvg[width=\linewidth]{../plots/cSoil_aus_anom.svg}
        \includegraphics[width=\linewidth]{../plots/cSoil_aus_anom.png}
    \end{subfigure}
    \begin{subfigure}[b]{0.4\linewidth}
        %\includesvg[width=\linewidth]{../plots/cSoil_aus_diff.svg}
        \includegraphics[width=\linewidth]{../plots/cSoil_aus_diff.png}
    \end{subfigure}
    \caption{Left: absolute carbon stocks. Right: difference between afforestation scenario and ssp585.}
    \label{fig:aus_region}
\end{figure}

\begin{figure}[H]
    \centering
    \includegraphics{../plots/regional_analysis_map.png}
    \caption{Box regions for regional analysis.}
    \label{fig:my_label}
\end{figure}

\begin{itemize}
    \item The esm-ssp585-ssp126Lu land use has bot afforestation and deforestation.
    \item Afforestation occurs in boreal Eurasia, boreal North America and Brazil.
    \item Afforestation and deforestation occurs in eastern North America, Central Africa and East Asia.
    \item Deforestation occurs in western Eurasia.
    \item Low afforestation/deforestation occurs in Australia.
    \item No appreciable difference in climate variables.
    \item Increases in cVeg for all regions except boreal North America, where some ensemble members show decreases in cVeg by the end of the century.
    \item Decreases in cSoil for Australia, Amazonia, Eastern North America. Western Eurasia, East Asia, Boreal Eurasia, Boreal North America, generally also show decreases in cSoil but there are a few ensemble members that show increases by the end of the century.
\end{itemize}

\begin{figure}[H]
    \centering
    \begin{subfigure}[b]{0.4\linewidth}
        %\includesvg[width=\linewidth]{../plots/cVeg_aus_anom.svg}
        \includegraphics[width=\linewidth]{../plots/cVeg_amazonia_diff.png}
    \end{subfigure}
    \begin{subfigure}[b]{0.4\linewidth}
        %\includesvg[width=\linewidth]{../plots/cVeg_aus_diff.svg}
        \includegraphics[width=\linewidth]{../plots/tas_amazonia_diff.png}
    \end{subfigure}
    \begin{subfigure}[b]{0.4\linewidth}
        %\includesvg[width=\linewidth]{../plots/cVeg_aus_anom.svg}
        \includegraphics[width=\linewidth]{../plots/cVeg_easternnorthamerica_diff.png}
    \end{subfigure}
    \begin{subfigure}[b]{0.4\linewidth}
        %\includesvg[width=\linewidth]{../plots/cVeg_aus_anom.svg}
        \includegraphics[width=\linewidth]{../plots/tas_easternnorthamerica_diff.png}
    \end{subfigure}
    \begin{subfigure}[b]{0.4\linewidth}
        %\includesvg[width=\linewidth]{../plots/cVeg_aus_diff.svg}
        \includegraphics[width=\linewidth]{../plots/cVeg_eastasia_diff.png}
    \end{subfigure}
    \begin{subfigure}[b]{0.4\linewidth}
        %\includesvg[width=\linewidth]{../plots/cVeg_aus_diff.svg}
        \includegraphics[width=\linewidth]{../plots/tas_eastasia_diff.png}
    \end{subfigure}
    \begin{subfigure}[b]{0.4\linewidth}
        %\includesvg[width=\linewidth]{../plots/cVeg_aus_diff.svg}
        \includegraphics[width=\linewidth]{../plots/cVeg_borealnorthamerica_diff.png}
    \end{subfigure}
    \begin{subfigure}[b]{0.4\linewidth}
        %\includesvg[width=\linewidth]{../plots/cVeg_aus_diff.svg}
        \includegraphics[width=\linewidth]{../plots/tas_borealnorthamerica_diff.png}
    \end{subfigure}
    \begin{subfigure}[b]{0.4\linewidth}
        %\includesvg[width=\linewidth]{../plots/cVeg_aus_diff.svg}
        \includegraphics[width=\linewidth]{../plots/cVeg_centralafrica_diff.png}
    \end{subfigure}
    \begin{subfigure}[b]{0.4\linewidth}
        %\includesvg[width=\linewidth]{../plots/cVeg_aus_diff.svg}
        \includegraphics[width=\linewidth]{../plots/tas_centralafrica_diff.png}
    \end{subfigure}
    \caption{ASD}
    \label{fig:aus_region}
\end{figure}

\subsection{Model inter-comparison}

\begin{figure}[H]
    \centering
    %\includesvg[width=\linewidth]{../plots/cVeg_aus_anom.svg}
    \includegraphics[width=\linewidth]{../plots/models_afforestation.png}
    \caption{cVeg in esm-ssp585-ssp126Lu scenario for 3 models.}
    \label{fig:aus_region}
\end{figure}

\subsection{Notes/Discussion}

\begin{itemize}
    \item The lack of climate warming mitigatino potential in the afforestation scenario may be due to redistribution of the global carbon budget into the natural sinks.
        An increase in vegetation uptake would lead to a decrease in atmospheric CO2 accumulation, which would result in a reduction of partial CO2 pressure on the ocean and hence lead to a reduction of CO2 in the ocean sink.
    \item This has bee seen in model simulations of the MPI-ESM simulations of afforestation. (See Hao-wei Wey).
    \item The biogeophysical impacts of growing trees could also warm the planet, especial in the higher latitudes where trees tend to be slow growing and the bare ground either has a high albedo or is covered in snow. This warming might offset the climate mitigation potential of afforestation.
\end{itemize}

\section{Conclusion?}

\printbibliography

\end{document}

