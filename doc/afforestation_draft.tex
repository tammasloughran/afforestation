% Draft manuscript on forestation in CMIP6 to be submitted to ESD? or similar journal.
% Note: Figures are plotted using .png files for now since SVG rendering is very
% slow for LaTeX. At a later date I will change the figures to .svg files. For example:
%\includesvg[width=\linewidth]{../plots/cVeg_aus_anom.svg}
% Journal candidates:
% - Earth system dynamics
% Preamble.
\documentclass[]{article}
\usepackage[a4paper, total={170mm,257mm}, left=20mm, top=20mm,]{geometry}
\usepackage{svg} % Use SVGs only for final draft.
\usepackage{float} % Allows the [H] option meaning "place figure here and only here"
\usepackage[section]{placeins} % Defines /FloatBarrier so that figures can't move.
\usepackage{subcaption} % Allows subfigures and captions
\usepackage{lineno} % Enable line numbers.
\usepackage{textcomp}
\usepackage[style=authoryear]{biblatex}
\usepackage{hyperref}
\usepackage{amsmath}
\addbibresource{afforestation_references.bib}
%\setlength{\parindent}{1cm}
\linenumbers
\title{Limited mitigation potentials of forestation under a high emissions scenario: multi-model and single model ensembles \footnote{I'm bad at titles. Someone think of something punchy please.}}
\author{Tammas Loughran et al.}

\begin{document}

\maketitle

\begin{center}
    \Large
    \vspace{0.9cm}
    \textbf{Abstract}
\end{center}

Forestation is a major component of future long-term emissions reduction and CO$_2$ removal strategies, but the viability of carbon stored in vegetation under future climates is highly uncertain.
We analyse the results from 7 CMIP6 models of a scenario with high fossil fuel emissions and forest expansion.
This scenario demonstrates the ability of forestation strategies to mitigate climate change under continued increasing CO$_2$ emissions and includes the potential impacts of increased CO$_2$ concentration and a warming climate on vegetation growth.
Model intercomparison shows that relying only on forestation as a CO$_2$ removal strategy has limited impact on global climate under a high global warming scenario, despite generating a substantial long-term carbon sink.
By using a single-model initial condition ensemble, we also show that there may be local increases in warm extremes in response to afforestation, particularly for decreases in the number of cool days.
Furthermore, we find evidence of shift in the global carbon balance, whereby increased uptake of carbon on the land of $\sim$25 PgC by 2100 and a decrease in the uptake of carbon by the ocean.

\raggedright
\parindent=.35in % Setting raggedright removes paragraph indents. This puts them back.

\section{Introduction}

Forests cover approximately 31\% of the global land surface \parencite{fao_global_2020} and are responsible for the removal of 30\% of total anthropogenic emissions from the atmosphere (\cite{friedlingstein_global_2022}).
Forestation is therefore often thought of as a viable strategy to remove CO$_2$ from the atmosphere and mitigate global warming \parencite{smith_long-term_2022}.
To effectively limit global warming to below 2 \textcelsius{} consistent with the Paris Agreement, there must be not only a reduction of fossil fuel emissions, but also CO$_2$ removal to offset industrial and agricultural activities that are difficult to decarbonize.
There must also be a shift in the reliance of global diets away from agricultural practices that emit greenhouse gases into the atmosphere (via e.g., forest clearing or ruminating herbivores).
This shift typically implies widespread abandonment of agricultural lands and an increase in natural vegetation, either as reforestation when degraded forests are restored or as afforestation of lands that were not previously occupied by forest.

Forestation and deforestation affect the climate in two main ways.
Firstly, by biogeochemical effects, i.e., changes to the global carbon cycle and carbon storage pools that impact the radiative absorption of the atmosphere.
And secondly by biogeophysical effects, i.e., changes in the physical properties of the land surface such as albedo and surface fluxes (\cite{bala_combined_2007}).
In general, forestation causes a global cooling biogeochemical effect as carbon is taken from the atmosphere and stored in vegetation and soils.
However, the biogeophysical impacts of forestation are more varied, with a cooling effect in the low--mid latitudes via enhanced evapotranspiration and further increased cloud cover reflecting incoming shortwave radiation.
In the mid--high latitudes, the biogeophysical impacts are a net warming effect because of the decrease of albedo from dark vegetation relative to the surfaces that are seasonally covered in snow.

Historically, there has been substantial deforestation in temperate forests of Eurasia and North America and in the last few decades deforestation has been focused on the tropics (\cite{goldewijk_estimating_2001}).
The net effect of deforestation is to cool the climate globally due to an increase in albedo  (\cite{davin_climatic_2010}).
\cite{boysen_global_2020} show a cooling of -0.22$\pm$0.2 \textcelsius{} among 9 climate models in idealized simulations.
\cite{hong_impacts_2022} also show that under a future deforestation scenario this cooling effect might reduce the incident of hot extremes by 5.5--0.9\%. \footnote{Still not sure which stat from this paper to include here.}
However, the effect of future afforestation may not necessarily be merely the inverse of the effect of future deforestation.

There have been a number of previous studies on the potential of forestation to remove CO$_2$ from the atmosphere, each using different methods of quantification.
For example, \cite{lenton_radiative_2009} used a simplified ``back-of-the-envelope" analytical calculation approach to estimate the radiative forcing effect of forestation, finding that it has substantial potential relative to most other geoengineering methods.
They estimate a removal of 73 Pg(C) by forestation can result in a decrease of 0.37 W m$^{-2}$ radiative forcing by 2100.
However, such simplified estimates assume that the land carbon storage is stable on long time scales, the decay of which under a future climate could only be estimated with an earth system model.

More sophisticated modelling studies better represent the complexity of the net effects of biogeophysical and biogeochemical impacts.
For example, \cite{sonntag_reforestation_2016} quantified the CO$_2$ removal potential using and Earth System Model, where a high CO$_2$ emissions scenario is simulated (taken from RCP8.5) in combination with the ``forestation" land-use from a low emissions scenario (taken from RCP4.5).
This previous study used the concentration driven CMIP5 (Coupled Model Intercomparison Project Phase 5) version of the MPI-ESM and the CMIP5 representative concentration pathways.
They find that a decrease in 85 ppm (215 Pg(C)) of atmospheric CO$_2$ concentrations by 2100 from forestation results in a cooling of 0.27 K globally (also reiterated in \cite{sonntag_quantifying_2018}).

Some experimental designs are more idealized and represent a more extreme deployment of forestation, where a huge portion of non-forested lands are forested.
\cite{de_hertog_biogeophysical_2022}, for example conducted an experiment with where all non-forested lands except bare ground are forested in a checker-board pattern.
While this is unfeasible in the real world, it allows for diagnosis of local and non-local effects of forestation.
The models here show that the local biogeophysical effects forestation produce a cooling in the tropics, while the non-local effects are large scale warming from forestation. 

So far, a multi-model intercomparison of the potential of forestation to remove CO$_2$ and mitigate climate change under a realistic scenario is absent from the literature.
This study aims to quantify the CO$_2$ removal potential of the forestation strategy using CMIP6 models, both at the global and regional scale.
The CMIP6 experiments present a standard set of experiments to draw on for understanding model uncertainty in the climate mitigation potential of forestation.
In particular, ACCESS-ESM1.5 contributed multiple single model ensembles to these experiments allowing us to examine the climate variability uncertainties from a single model.

\section{Methods}

\subsection{Experiments}

The CMIP6 experimental design specifies a set of standard simulations called the Diagnostic, Evaluation and Characterization of Klima (DECK) \parencite{eyring_overview_2016}, which are typically used as the foundation to study more specific research questions.
The DECK includes a pre-industrial control simulation (\textit{piControl}) with constant greenhouse gas concentration forcing, and a historical (\textit{historical}) simulation with transient historical greenhouse gas concentration forcing.
Furthermore, the DECK specifies corresponding simulations (\textit{esm-piControl} and \textit{esm-hist}) that are run in full ``Earth system model" mode with a fully interactive dynamical carbon cycle and for the historical period, fossil fuel and industrial greenhouse gas emissions forcings.
In order to capture the carbon cycle and dynamics of the entire Earth system in response to forestation, including biogeochemical and biogeophysical feedbacks of the land surface and climate system, we use only the Earth system model simulations from relevant intercomparison projects. These emissions-driven simulations are initialized at 2015 from the end of the \textit{esm-hist} experiment.

The forestation scenario we use is the Land Use Model Intercomparison Project's \textit{esm-ssp585-ssp126Lu}, spanning the years 2015--2100 \parencite{lawrence_land_2016}.
This simulation features the surface fossil fuel related CO$_{2}$ emissions from the SSP5-8.5 scenario, and the land-use change from the SSP1-2.6 scenario \parencite{oneill_scenario_2016}.
The SSP5-8.5 scenario emissions represent a high emissions future pathway that results in a radiative forcing of 8.5 W m$^{-2}$ in 2100.
The SSP1-2.6 scenario land-use change assumes a future of sustainable development \parencite{van_vuuren_energy_2017}.
This means low population growth, low pressures on land-use, environmentally friendly changes in diets and increased agricultural efficiency and yields, resulting in abandonment of agricultural lands.
This allows for the expansion of natural lands and forest cover.
We use this scenario because it represents a plausible future scenario that would provide a lower bound on the survivability of vegetation and CO$_2$ draw-down potential of the land surface under a warmer climate.

The reference simulation we use to compare to \textit{esm-ssp585-ssp126Lu} is the \textit{esm-ssp585} from the Coupled Climate-Carbon Cycle Model Intercomparison Project (C4MIP) \parencite{jones_c4mip_2016}.
This is the emissions-driven Earth system model simulation corresponding to the SSP5-8.5 high greenhouse gas scenario that assumes that development is driven by fossil fuels \parencite{oneill_scenario_2016}.
In general, the land use change in this scenario has less expansion of forested lands than \textit{esm-ssp585-ssp126Lu}, but is a mixture of moderate forestation and deforestation.
The notable difference in land-use change compared the forestation scenario is the deforestation in the central African region (Figure S\ref{fig:ACCESS_land_cover}).
By taking the difference between the forestation simulation and the \textit{esm-ssp585} simulation (Equation \ref{equ:diff}), we can examine the impact of forestation on the climate and the carbon cycle.

\begin{equation}
    X|_{for} = X|_{esm-ssp585-ssp126Lu} - X|_{esm-ssp585}
    \label{equ:diff}
\end{equation}

The land-use change dataset used in all CMIP6 experiments is the Land-use Harmonization dataset version 2 (LUH2; \cite{hurtt_harmonization_2020}).
There are only a few areas where there is deforestation in the \textit{esm-ssp585-ssp126Lu} experiment.
These occur in deciduous broad leaf forests in eastern North America, China and western Russia (Figure \ref{fig:land_use_map}).
Furthermore, it is important to acknowledge that the LUH2 SSP1-2.6 scenario underestimates the tree cover that is originally dictated by IMAGE (the integrated assessment model that produces the SSP1-2.6 scenario).
This is due to differences in the definition of tree cover in the integrated assessment models as well as the effects of harmonization of that data with observed historical cover fractions. \footnote{How does this affect the inferences on the maximum potential of forestation from this scenario? I.e., would we therefore underestimate the mitigation potential?}

\begin{figure}[H]
    \centering
    \begin{subfigure}[b]{\linewidth}
        \includegraphics[width=\linewidth]{../plots/tree_frac_deltat.png}
    \end{subfigure}
    \caption{Change in tree fraction from the Land-use Harmonization 2 SSP1-2.6 scenario between the year 2100 and 2015.}
    \label{fig:land_use}
\end{figure}

\subsection{Participating models}

There are 7 \footnote{NorESM5 contributed a simulation to LUMIP for the \textit{esm-ssp585-ssp126Lu} experiment, however we have excluded it since it was run in concentration driven mode.} Earth system models that participated in both LUMIP and C4MIP with simulations available for \textit{esm-ssp585-ssp126Lu} and \textit{esm-ssp585}.
A brief overview of these models is presented in Table \ref{tab:models}.
2 models included wildfire schemes, 3 models included dynamical vegetation, and 6 models included nitrogen nutrient limitation.
ACCESS-ESM1-5 is the only model to include phosphorus limitation.
Furthermore, ACCESS-ESM1-5 is the only model to have multiple ensemble members available for both simulations, of which there are 10 for each experiment.

\begin{table}[H]
    \centering
    \begin{tabular}{lllllllll}
        \hline
Model         & Fire & N   & P   & Dyn. & PFTs & Nat. PFT & Res. & Ref.                                         \\ \hline
ACCESS-ESM1-5 & No   & Yes & Yes & No    & 10       & 9       & 1.8758×1.258°   & \cite{ziehn_australian_2020}   \\
BCC-CSM2-MR   & No   & Yes & No  & No    & 14       & 11      & 110×110km       & \cite{li_development_2019}     \\
CanESM5       & No   & No  & No  & No*   & 9        & 7       & $\sim$2.8×2.8°       & \cite{swart_canadian_2019}     \\
GFDL-ESM4     & Yes  & No  & No  & No    & 7        & 5       & 100×100km       & \cite{dunne_gfdl_2020}         \\
MIROC-ES2L    & No   & Yes & No  & No    & 14       & 13      & 2.8×2.8°        & \cite{hajima_development_2020} \\
MPI-ESM1-2-LR & Yes  & Yes & No  & Yes   & 13       & 12      & 1.9°            & \cite{giorgetta_climate_2013}  \\
%NorESM2-LM    & Yes  & Yes & No  & No    & 22       & 14      & 2×2°            & \cite{seland_norwegian_2020}   \\
UKESM1-0-LL   & No   & Yes & No  & Yes   & 12       & 8       & 1.25×1.875°     & \cite{sellar_ukesm1_2019}      \\ \hline
    \end{tabular}
    \caption{Models participating in both Land-Use Model Intercomparison Project and Coupled Climate-Carbon model Intercomparison Project with available simulations for \textit{esm-ssp585-ssp126Lu} and \textit{esm-ssp585}. The columns indicate whether the models land surface component have representations of wildfire, nutrients, dynamical vegetation and the number of plant functional types. Natural plant functional types excludes agricultural crops and pasture. *Can-ESM5 does have plant dynamics in the sense of plant demographics but not dynamical plant type competition.}
    \label{tab:models}
\end{table}

\subsection{Regional analysis}

Regional analyses have been divided into 8 regions that feature substantial changes in tree cover.
These regions are shown in Figure \ref{fig:land_use}.
The temperature and precipitation values over the ocean are masked out to represent land only changes in these variables.
The largest differences in tree cover between the two simulations occurs in various hot spots around the globe.
To examine the local scale impact of large changes in tree cover on temperature and precipitation, histograms of the frequency distributions for specific grid-points are calculated, shown by the dots in Figure \ref{fig:land_use}.
Here, we focus only on ACCESS-ESM1-5, since it is the only model that contributed 10 ensemble members, and thus the statistics of the distribution are most robust for this model.

%\begin{figure}[H]
%    \centering
%    \includegraphics{../plots/regional_analysis_map.png}
%    \caption{Box regions for regional analysis. Dots indicate grid-points that feature large transitions to forest cover and are used in the analysis of the temperature distributions in each simulation.}
%    \label{fig:box_regions}
%\end{figure}

\subsection{Statistical methods}

We calculate trends in the difference of global mean temperatures and precipitation to examine the change in temperature as forests expand.
For this we use the Theil-Sen slope estimator and test its significance at the 5\% level using the Mann-Kendall trend test. 

We also calculate probability histograms to examine the distribution of daily temperatures for the ACCESS-ESM1-5.
The difference in the temperature distribution in the two simulations in response to forestation is tested using the 2-sample Kolmogorov-Smirnov test for the equality of distributions at the 5\% level.

Finally, to examine the relationship of forestation on climate, correlations were done on ensemble mean surface air temperature and tree fraction using the Spearman rank correlation and the significance was tested at the 5\% level \parencite{kokoska2000crc}.

\section{Results and discussion}

\subsection{Global Multi-model Inter-comparison}

\subsubsection{Carbon cycle}

Each model has a unique internal representation of woody tree biomes and pre-industrial potential vegetation.
It follows that each model interprets the allocation of natural lands from LUH2 to model-specific plant functional types (PFTs) differently.
Combined with differences in grid resolution, this results in a variety of changes in global tree cover for each model (Fig. \ref{fig:land_use_map}).
UKESM1-0-LL, CanESM5 and MPI-ESM1-2-LR include vegetation dynamics that respond to changes in climate, and these are the models that deviate from the LUH2 forcing the most.
CanESM5 in particular stands out as having a net loss of tree area by 2100.
MPI-ESM features substantial Amazonian die-back, but large amounts of forest increase in semi-arid regions in Africa and Australia.

ACCES-ESM1-5 is an example that closely follows the spatial distribution of the LUH2 land-use forcing.
By 2100, ACCESS-ESM1-5 has a forest expansion of 1.59 million km² and agricultural abandonment of 1.11 million km$^2$.
By mid-century, crops reach a minimum of 2.74 million km² less than in 2015, before rising again in the latter half of the century (Figure \ref{fig:land_use_map}).
Forestation is dominated by growth of evergreen broad leaf forests, followed by evergreen needle leaf forests, then deciduous broad leaf forests.
Deciduous needle leaf forests only make up a small fraction of forests and do not show any expansion.

\begin{figure}[H]
    \centering
    \begin{subfigure}[b]{0.9\linewidth}
        \includegraphics[width=\linewidth]{../plots/tree_area_ssp126_all_models_global_sum.png}
    \end{subfigure}
    \begin{subfigure}[b]{0.9\linewidth}
        \includegraphics[width=\linewidth]{../plots/tree_area_map_2015-2100_delta.png}
    \end{subfigure}
    \caption{Tree cover area changes between 2015 to 2100 in the forestation scenario for each model. Global mean tree cover fraction for each model and CABLE tree PFT areas for the SSP1-2.6 scenario. (Several models are missing the treeFrac variable.)}
    \label{fig:land_use_map}
\end{figure}

\begin{figure}[H]
    \centering
    \begin{subfigure}[b]{0.45\linewidth}
        \includegraphics[width=\linewidth]{../plots/cLand_model_intercomparison_diff.png}
    \end{subfigure}
    \begin{subfigure}[b]{0.45\linewidth}
        \includegraphics[width=\linewidth]{../plots/cVeg_model_intercomparison_diff.png}
    \end{subfigure}
    \begin{subfigure}[b]{0.45\linewidth}
        \includegraphics[width=\linewidth]{../plots/cSoil+cLitter_model_intercomparison_diff.png}
    \end{subfigure}
    \caption{cLand, cVeg and cLitter+cSoil in the \textit{esm-ssp585-ssp126Lu} scenario for 6 CMIP6 models. ACCESS-ESM is plotted as the ensemble mean and the blue shading indicates the ACCESS-ESM1-5 ensemble range.}
    \label{fig:models_cpools}
\end{figure}

CanESM5 has differing ensemble methods for \textit{esm-ssp585} and \textit{esm-ssp585-ssp126Lu}.
One former uses r1i1p1f1 and the latter uses r1i1p2f1.
Therefore, there is some additional perturbation applied to one of those experiments.
The CanESM5 model (green line in \ref{fig:models_cpools}) has been bias corrected by subtracting the difference in the carbon pools between the reference and forestation simulations at the start of the experiment (2015).
This brings CanESM5 in line with other models for total land carbon.
CanESM has net deforestation by 2100 but increases in cVeg in the middle half of the century, which are transferred to the litter and soil carbon pools by 2100.
It also has no significant change in global temperature by 2100.

ACCESS-ESM and UKESM are approximately in the middle of the model spread.
These models share the same atmospheric model component and therefore share many of the same climate physics, however UKESM has a greater transient climate response to forestation and atmospheric CO$_2$ changes.

For vegetation related pools (cLand and cVeg) models tend to diminish in productivity by 2100.
The models  either maintain a draw-down potential of $\sim$20--30PgC by 2100, or the gains from the middle of the century have been lost by 2100.
Soil carbon shows varied response to forestation.
ACCESS and UKESM show decreases in soil carbon, while other models show increases.

\begin{figure}
    \centering
    \includegraphics[width=\linewidth]{../plots/models_co2_diff.png}
    \caption{Difference in atmospheric CO2 concentration between \textit{esm-ssp585} and \textit{esm-ssp585-ssp126Lu}}
    \label{fig:models_CO2}
\end{figure}

The impact of an increased land surface sink is a corresponding decrease in atmospheric CO$_2$ concentrations as demonstrated by (Fig. \ref{fig:models_CO2}).
The largest change in concentration is $\sim$22 ppm from MPI-ESM1-2-LR, which is much lower than the 85 ppm decrease in the scenario used by \cite{sonntag_reforestation_2016}.
As we will see in the following section, the impact of this on the climate is expected to be 
Strangely, BCC-CSM2-MR shows increased CO$_2$ concentrations in the middle of the century when the land sink is at its maximum, but approximately no change in atmospheric CO$_2$ by 2100, consistent with the zero net uptake of land carbon at the time.
The ACCESS-ESM1-5 ensemble range demonstrates that atmospheric CO$_2$ is strongly sensitive to internal climate variability.

\subsubsection{Climate response}

The change in global surface air temperature in response to atmospheric CO$_2$ concentrations and surface albedo is small in most models.
The trend in the difference of the afforestation and reference simulations over the course of the century is not significant in all but one model (Fig. \ref{fig:models_tas_trends}).
BCC-CSM2-MR shows a weak but significant decreasing trend in response to forestation.
Furthermore, there is some disagreement in the sign of the change in global temperature among models and ensemble members, with CanESM5 showing a non-significant positive trend.
The effect that internal climate variability can have on the trends is demonstrated by the ACCESS-ESM1-5 ensemble.
While the ensemble mean trend showed no significant change, 3 members showed a significant positive trend.
Similar to global mean surface air temperature, the response of global mean precipitation to forestation is also unclear from the models (Fig. S\ref{fig:models_pr_trends}).
All models show no significant trends in global precipitation rate.

\begin{figure}[H]
    \centering
    \includegraphics[width=\linewidth]{../plots/tas_trends.png}
    \caption{Surface air temperature trends in the difference between the forestation scenario and the reference simulation. Solid lines indicate trends that are statistically significant at the 5\% level, and dotted lines are not significant trends. (+) and (-) denote the sign of the trend.}
    \label{fig:models_tas_trends}
\end{figure}

\subsection{Regional land carbon and climate responses in ACCESS-ESM1.5}

\subsubsection{Overview of ACCESS-ESM1.5 response to forestation}

Since ACCESS-ESM1-5 has 10 ensemble members available, and the regional distribution of new forest growth varies greatly among the models, the regional analysis will focus only on ACCESS-ESM1-5.
The single model ensemble allows us to examine the impact of forestation on the full distribution of regional surface temperatures and carbon uptake.
The carbon cycle in ACCESS-ESM1-5 in the forestation scenario reflects the temporal pattern of forestation in the forcing data well (Fig. \ref{fig:global_carbon_budget}a), with the land surface acting as a sink in the first half of the century when most of the forestation occurs and becoming a weak source towards the end of the century.
The ocean sink strengthens as the partial pressure of CO$_2$ on the ocean surface increases throughout the century from increasing atmospheric CO$_2$ concentrations.
However, the lower atmospheric CO$_2$ concentrations relative to the reference simulation results in the ocean absorbing cumulatively $\sim$1.3±0.5 Pg C less by 2100 (Fig. \ref{fig:global_carbon_budget}b).
Globally, cLand increases by 1.3±0.3\% of the cLand in the reference simulation, with 3.3±0.4 increase in cVeg and 0.5±3 decrease in cSoil.
ACCESS-ESM1-5 is also the only model to include phosphorus nutrient limitation, and therefore the ACCESS-ESM1-5 cVeg pools contrasts to other models, reaching a stable limit by 2100 while other model's cVeg are still increasing by 2100 (Fig. S\ref{fig:models_absolute}b).
The climate in both the forestation and reference simulations are similar, with a warming of $\sim$4 \textcelsius{} by 2100 and precipitation increases by 0.216 kg m$^{-2}$ day$^{-1}$.

\begin{figure}[H]
    \centering
    \begin{subfigure}[b]{0.4\linewidth}
        \includegraphics[width=\linewidth]{../plots/esm-ssp585-ssp126Lu_budget.png}
    \end{subfigure}
    \begin{subfigure}[b]{0.4\linewidth}
        \includegraphics[width=\linewidth]{../plots/fgco2_ocean_carbon_aff_ssp585_cumulative.png}
    \end{subfigure}
    \caption{Left: Difference in atmospheric CO$_2$ concentration between the forestation scenario and ssp585. Shading indicates the ensemble range. Right: ensemble mean of carbon flux budget in the forestation scenario. Difference between the forestation scenario and ssp585 for the cumulative downward ocean carbon mass flux. Shading indicates the ensemble range.}
    \label{fig:global_carbon_budget}
\end{figure}

%\begin{table}[H]
%    \centering
%    \begin{tabular}{@{}llll@{}}
%    \hline
%        & $\Delta$ Carbon (Pg) & \% present day & \% of 2080-2100 ssp585 \\ \hline
%cLand   & 22.0±5.3    & 1.3±0.3        & 1.3±0.3                \\
%cVeg    & 24.8±2.7    & 3.7±0.4        & 3.3±0.4                \\
%cLitter & 0.4±0.3     & 0.8±0.6        & 0.9±0.7                \\
%cSoil   & -4.3±2.3    & -0.5±0.3       & -0.5±0.3               \\ \hline
%    \end{tabular}
%    \caption{C pools expressed as a percentage of the present day carbon content or  as a percentage of the end of \textit{esm-ssp585}. This information can mostly be seen in Fig. 4 and is only presented here to show the options on how to present the data. Will likely be removed in favour of quoting in text.}
%    \label{tab:cpools_table}
%\end{table}

\subsubsection{Regional changes in vegetation and climate extremes}

There are increases in cLand for all regions and ensemble members except Amazonia, where some ensemble members show a small decrease in cVeg by 2100, due to internal climate variations.
The region with the largest change in cLand is Central Africa (Fig. \ref{fig:accesss_regional_cland}e), however this change is due to avoided deforestation that occurs in the reference simulation, rather than due to new forest growth in the forestation experiment (Fig. S\ref{fig:ACCESS_land_cover}a).
Nevertheless, this highlights the importance of including avoided deforestation in future long-term national climate strategies, since a considerable portion of land-use emissions comes from the loss of additional sink capacity from deforestation \parencite{gitz_amplifying_2003, pongratz_terminology_2014, obermeier_modelled_2021}.
The increased CO$_2$ draw-down from the combined effect of forestation and CO$_2$ fertilization are partially offset by the increase in respiration in the litter and soil (Fig. \ref{fig:access_cpools}), particularly in Australia where there is no increase in forest cover in the SSP1-2.6 scenario.

\begin{figure}[H]
    \centering
    \begin{subfigure}[b]{0.4\linewidth}
        \includegraphics[width=\linewidth]{../plots/cLand_amazonia_diff.png}
    \end{subfigure}

    \begin{subfigure}[b]{0.4\linewidth}
        \includegraphics[width=\linewidth]{../plots/cLand_easternnorthamerica_diff.png}
    \end{subfigure}
    \begin{subfigure}[b]{0.4\linewidth}
        \includegraphics[width=\linewidth]{../plots/cLand_eastasia_diff.png}
    \end{subfigure}
    \begin{subfigure}[b]{0.4\linewidth}
        \includegraphics[width=\linewidth]{../plots/cLand_borealnorthamerica_diff.png}
    \end{subfigure}
    \begin{subfigure}[b]{0.4\linewidth}
        \includegraphics[width=\linewidth]{../plots/cLand_centralafrica_diff.png}
    \end{subfigure}
    \caption{Experiment differences of cLand for each region in Fig. \ref{fig:land_use}.}
    \label{fig:accesss_regional_cland}
\end{figure}

The relationship of surface air temperature and changes in total tree cover fraction varies by region.
The correlation of temperature and tree cover fraction is positive in the tropical regions of Central Africa, South America, the Maritime Continent, and East Asia (Fig. \ref{fig:map_tas_tree_correlation}).
These correlations show high significance, with more than half the ensemble members agreeing that the correlations are significant at the 5\% level.
Hence, increased tree cover fraction increases surface air temperatures and the effect of decreased surface albedo dominates.
In contrast, the Central African region that surrounds the avoided deforestation in the reference simulations shows the opposite effect, whereby increased tree cover negatively correlates with air temperature and hence the cooling effect of evapotranspiration dominates.
Furthermore, in the sub-tropcal and boreal regions of North America, Europe and Boreal Eurasia the correlation is negative, indicating that as forest cover increases, temperature decreases.
However, these correlations show much less significance with only 2--3 ensemble members being significantly correlated.

The biogeophysical impacts of growing trees cause localized warming at the extreme ends of the temperature distribution, especially in the higher latitudes where trees tend to be slow growing and the ground a high albedo from snow cover.
This warming may offset the biogeochemical climate mitigation potential of forestation.
For specific grid-points with large changes from grass to tree biomes, the distribution of summer daily maximum surface air temperature for both the forestation and reference simulations are shown in Figure \ref{fig:tasmax_distribution}.
The Amazon grid-point features changes in mostly C4 grass to Evergreen broad leaf forest, representing an increase in tree cover fraction of 60\%.
This corresponds to a statistically significant change in the distribution, particularly for temperatures greater than 50 \textcelsius{} (Fig. \ref{fig:tasmax_distribution}b).

The increase in the high end of the distribution in response to forestation is not consistent for all regions.
For example, the large increase in forest cover for the grid point in Asia corresponds to a decrease in temperatures greater than 23 \textcelsius{}, while temperatures 17--23 \textcelsius{} increase (Fig. \ref{fig:tasmax_distribution}d).
The changes in daily maximum temperature at the lower end of the distribution in response to forestation are much more regionally consistent, showing a decrease in cooler than average days.

Some regions showed decreases in sirface air temperature in response to increasing tree cover.
Of particular not is Eastern North America which features a large transition from C3 crops to deciduous broadleaf (Fig. \ref{fig:tasmax_distribution} e).
The distribution shows decreases in the number of warm and cool days in the forestation experiment.
In ACCESS-ESM1-5, deciduous broadleaf forests have the highest reflectance of all the 

\begin{figure}[H]
    \centering
    \begin{subfigure}[b]{0.4\linewidth}
        \includegraphics[width=\linewidth]{../plots/pfts_amazongridpoint.png}
    \end{subfigure}
    \begin{subfigure}[b]{0.4\linewidth}
        \includegraphics[width=\linewidth]{../plots/tasmax_histogram_amazongridpoint.png}
    \end{subfigure}
    \begin{subfigure}[b]{0.4\linewidth}
        \includegraphics[width=\linewidth]{../plots/pfts_asiagridopint.png}
    \end{subfigure}
    \begin{subfigure}[b]{0.4\linewidth}
        \includegraphics[width=\linewidth]{../plots/tasmax_histogram_asiagridopint.png}
    \end{subfigure}
    \begin{subfigure}[b]{0.4\linewidth}
        \includegraphics[width=\linewidth]{../plots/pfts_northamericagridpoint.png}
    \end{subfigure}
    \begin{subfigure}[b]{0.4\linewidth}
        \includegraphics[width=\linewidth]{../plots/tasmax_histogram_northamericagridpoint.png}
    \end{subfigure}
    \caption{Left: Changes in plant functional types for select grid points in \textit{esm-ssp585-ssp126Lu} in ACCES-ESM1-5. Right: Distributions of summer time (June--August or December--February) maximum daily temperature for the last 20 years of \textit{esm-ssp585-ssp126Lu} (green) and \textit{esm-ssp585} (yellow).}
    \label{fig:tasmax_distribution}
\end{figure}

\begin{figure}[H]
    \centering
    \begin{subfigure}[b]{0.5\linewidth}
        \includegraphics[width=\linewidth]{../plots/correlation_tree_tas.png}
    \end{subfigure}

    \begin{subfigure}[b]{0.5\linewidth}
        \includegraphics[width=\linewidth]{../plots/correlation_tree_tas_significance.png}
    \end{subfigure}
    \caption{Correlation of 2m surface air temperature (\textit{esm-ssp585-ssp126Lu} \textit{esm-ssp585} difference) with tree fraction in ACCESS-ESM1-5, as wel as the number of significant correlations at the 5\% level.}
    \label{fig:map_tas_tree_correlation}
\end{figure}

\section{Concluding Remarks}

In this study, we conduct a multi-model intercomparison of a plausible scenario for forestation as a means of carbon dioxide removal.
Models show a diverse interpretation of the forestation spatial patterns and as a result show a large range of outcomes for long term carbon storage in forests.
4 models show that a stable but limited carbon sink by 2100, while 3 models show that the mitigation gains from forestation are completely lost by 2100.

The models show no significant impact of forestation on global climate under a high emissions and high global warming scenario.
Forestation causes a shift in the global carbon balance, whereby increased uptake of carbon on the land of $\sim$25 Pg C by 2100 results in a decrease in the uptake of carbon by the ocean.
There are some increases in local scale warm extremes in locations where forestation occurs.
Grid-points that have strong forestation show decreases in the number of cool days and increase in the number of extreme hot days.
Forest management strategies may also provide the potential to grow trees that have a higher albedo, which could offset this effect.
However, this should be addressed in future research.

Forestation and forest management provides additional co-benefits such as increased habitat, biodiversity and decreased runoff and soil erosion.
However, many of these features are not yet simulated in Earth system models.
While it is clear that most models show that growing trees under a warmer (and sometimes drier) climate is not a complete solution to climate mitigation, this does not mean that forestation should not be done.
For forestation to be viable as a carbon dioxide strategy it must also exist in conjunction with other strategies.
By first regrowing forests for the purpose of carbon dioxide removal, forestation increases the natural land-based carbon and enables further development and supply of feedstock for other strategies. 
For example, none of the models in this study account for the potential of forests to be used for other carbon dioxide removal strategies such as for bioenergy with carbon capture and storage, and is a potential avenue for further research in the Earth System.
If some forests are sustainably harvested and regrown, then the removal of CO$_2$ from the atmosphere accumulates over time, and while vegetation may be lost in individual natural disturbance events such as fires, the historically removed carbon remains locked.

\printbibliography

\section{Supplementary Material}
\setcounter{figure}{0}

\subsection{PFTs}

\begin{figure}[H]
    \centering
    \begin{subfigure}[b]{0.45\linewidth}
        \includegraphics[width=\linewidth]{../plots/tree_frac_deltat_ssp585.png}
    \end{subfigure}
    \caption{Temporal anomaly in tree cover fractions in SSP5-8.5 scenario from LUH2.}
    \label{fig:LUH2_tree_frac_ssp585}
\end{figure}

\begin{figure}[H]
    \centering
    \begin{subfigure}[b]{0.4\linewidth}
        \includegraphics[width=\linewidth]{../plots/treeFrac_esm-ssp585_anomaly.png}
    \end{subfigure}
    \begin{subfigure}[b]{0.4\linewidth}
        \includegraphics[width=\linewidth]{../plots/grassFrac_esm-ssp585_anomaly.png}
    \end{subfigure}
    \begin{subfigure}[b]{0.4\linewidth}
        \includegraphics[width=\linewidth]{../plots/cropFrac_esm-ssp585_anomaly.png}
    \end{subfigure}
    \caption{Temporal anomaly in area of trees, grass and crop between 2015 and 2100 for \textit{esm-ssp585} in ACCESS-ESM1.5}
    \label{fig:ACCESS_land_cover}
\end{figure}

% Also put the SSP585 cover fraction maps for each model here.

\begin{figure}[H]
    \centering
    \begin{subfigure}[b]{0.45\linewidth}
        \includegraphics[width=\linewidth]{../plots/land_areas.png}
    \end{subfigure}
    \begin{subfigure}[b]{0.45\linewidth}
        \includegraphics[width=\linewidth]{../plots/land_areas_diff.png}
    \end{subfigure}
    \begin{subfigure}[b]{0.45\linewidth}
        \includegraphics[width=\linewidth]{../plots/CABLE_forests.png}
    \end{subfigure}
    \caption{Land use areas in ACCESS-ESM1.5. Left: ssp585 (dashed) and forestation scenario (solid). Right: difference between ssp585 and forestation scenario.}
    \label{fig:ACCESS_land_use}
\end{figure}

\begin{figure}[H]
    \centering
    \begin{subfigure}[b]{0.45\linewidth}
        \includegraphics[width=\linewidth]{../plots/CABLE_forests_deciduous_needle.png}
    \end{subfigure}
    \caption{Global change in deciduous needleleaf forest in ACCESS-ESM1-5.}
    \label{fig:ACCESS_dec_needle_cover}
\end{figure}

\subsection{Model intercomparison}

\begin{figure}[H]
    \centering
    \includegraphics[width=\linewidth]{../plots/pr_trends.png}
    \caption{Surface global precipitation rate trends in the difference between the forestation scenario and the reference simulaiton. Solid lines indicate trends that are statistically significant at the 5\% level, and dotted lines are not significant trends. `+' and `-' denote the sign of the trend.}
    \label{fig:models_pr_trends}
\end{figure}

\begin{figure}[H]
    \centering
    \begin{subfigure}[b]{0.45\linewidth}
        \includegraphics[width=\linewidth]{../plots/cLand_model_intercomparison_esm-ssp585-ssp126Lu.png}
    \end{subfigure}
    \begin{subfigure}[b]{0.45\linewidth}
        \includegraphics[width=\linewidth]{../plots/cVeg_model_intercomparison_esm-ssp585-ssp126Lu.png}
    \end{subfigure}
    \begin{subfigure}[b]{0.45\linewidth}
        \includegraphics[width=\linewidth]{../plots/cLitter_model_intercomparison_esm-ssp585-ssp126Lu.png}
    \end{subfigure}
    \begin{subfigure}[b]{0.45\linewidth}
        \includegraphics[width=\linewidth]{../plots/cSoil_model_intercomparison_esm-ssp585-ssp126Lu.png}
    \end{subfigure}
    \begin{subfigure}[b]{0.45\linewidth}
        \includegraphics[width=\linewidth]{../plots/tas_model_intercomparison_esm-ssp585-ssp126Lu.png}
    \end{subfigure}
    \begin{subfigure}[b]{0.45\linewidth}
        \includegraphics[width=\linewidth]{../plots/pr_model_intercomparison_esm-ssp585-ssp126Lu.png}
    \end{subfigure}
    \caption{Absolute values of the forestation scenario's global mean carbon pools, 2m surface air temperature and precipitation for each model. The solid line for ACCESS ESM-1-5 is the ensemble mean and the shading indicates the ensemble range.}
    \label{fig:models_absolute}
\end{figure}

\begin{figure}[H]
    \centering
    \begin{subfigure}[b]{0.45\linewidth}
        \includegraphics[width=\linewidth]{../plots/cLand_model_intercomparison_esm-ssp585.png}
    \end{subfigure}
    \begin{subfigure}[b]{0.45\linewidth}
        \includegraphics[width=\linewidth]{../plots/cVeg_model_intercomparison_esm-ssp585.png}
    \end{subfigure}
    \begin{subfigure}[b]{0.45\linewidth}
        \includegraphics[width=\linewidth]{../plots/cLitter_model_intercomparison_esm-ssp585.png}
    \end{subfigure}
    \begin{subfigure}[b]{0.45\linewidth}
        \includegraphics[width=\linewidth]{../plots/cSoil_model_intercomparison_esm-ssp585.png}
    \end{subfigure}
    \caption{Absolute values of the SSP5-8.5 scenario's global mean carbon pools, 2m surface air temperature and precipitation for each model. The solid line for ACCESS ESM-1-5 is the ensemble mean and the shading indicates the ensemble range.}
    \label{fig:models_absolute2}
\end{figure}

\subsection{ACCESS-ESM1-5 global pools and fluxes}

\begin{figure}[H]
    \centering
    \begin{subfigure}[b]{0.4\linewidth}
        \includegraphics[width=\linewidth]{../plots/cVeg_ACCESS-ESM1.5_esm-ssp585-ssp126Lu_ensembles_anomalies.png}
    \end{subfigure}
    \begin{subfigure}[b]{0.4\linewidth}
        \includegraphics[width=\linewidth]{../plots/cLitter_ACCESS-ESM1.5_esm-ssp585-ssp126Lu_ensembles_anomalies.png}
    \end{subfigure}
    \begin{subfigure}[b]{0.4\linewidth}
        \includegraphics[width=\linewidth]{../plots/cSoil_ACCESS-ESM1.5_esm-ssp585-ssp126Lu_ensembles_anomalies.png}
    \end{subfigure}
\begin{subfigure}[b]{0.4\linewidth}
        \includegraphics[width=\linewidth]{../plots/cLand_ACCESS-ESM1.5_esm-ssp585-ssp126Lu_ensembles_anomalies.png}
    \end{subfigure}
    \caption{ACCESS-ESM1-5 carbon pools. \textit{esm-ssp585-ssp126Lu} relative to the 2005--2025 period.}
    \label{fig:access_cpools}
\end{figure}

\begin{figure}[H]
    \centering
    \begin{subfigure}[b]{0.4\linewidth}
        \includegraphics[width=\linewidth]{../plots/gpp_ACCESS-ESM1.5_esm-ssp585-ssp126Lu_ensembles_anomalies.png}
    \end{subfigure}
    \begin{subfigure}[b]{0.4\linewidth}
        \includegraphics[width=\linewidth]{../plots/gpp_ACCESS-ESM1.5_esm-ssp585-ssp126Lu_ensembles_diff.png}
    \end{subfigure}
    \begin{subfigure}[b]{0.4\linewidth}
        \includegraphics[width=\linewidth]{../plots/npp_ACCESS-ESM1.5_esm-ssp585-ssp126Lu_ensembles_anomalies.png}
    \end{subfigure}
    \begin{subfigure}[b]{0.4\linewidth}
        \includegraphics[width=\linewidth]{../plots/npp_ACCESS-ESM1.5_esm-ssp585-ssp126Lu_ensembles_diff.png}
    \end{subfigure}
    \begin{subfigure}[b]{0.4\linewidth}
        \includegraphics[width=\linewidth]{../plots/ra_ACCESS-ESM1.5_esm-ssp585-ssp126Lu_ensembles_anomalies.png}
    \end{subfigure}
    \begin{subfigure}[b]{0.4\linewidth}
        \includegraphics[width=\linewidth]{../plots/ra_ACCESS-ESM1.5_esm-ssp585-ssp126Lu_ensembles_diff.png}
    \end{subfigure}
    \begin{subfigure}[b]{0.4\linewidth}
        \includegraphics[width=\linewidth]{../plots/rh_ACCESS-ESM1.5_esm-ssp585-ssp126Lu_ensembles_anomalies.png}
    \end{subfigure}
    \begin{subfigure}[b]{0.4\linewidth}
        \includegraphics[width=\linewidth]{../plots/rh_ACCESS-ESM1.5_esm-ssp585-ssp126Lu_ensembles_diff.png}
    \end{subfigure}
    \begin{subfigure}[b]{0.4\linewidth}
        \includegraphics[width=\linewidth]{../plots/nbp_ACCESS-ESM1.5_esm-ssp585-ssp126Lu_ensembles_anomalies.png}
    \end{subfigure}
    \begin{subfigure}[b]{0.4\linewidth}
        \includegraphics[width=\linewidth]{../plots/nbp_ACCESS-ESM1.5_esm-ssp585-ssp126Lu_ensembles_diff.png}
    \end{subfigure}
    \caption{Carbon fluxes in \textit{esm-ssp585-ssp126Lu}.  Left: relative to 2005--2025. Right: relative to \textit{esm-ssp585}.}
    \label{fig:access_cflux}
\end{figure}

\begin{figure}[H]
    \centering
    \begin{subfigure}[b]{0.45\linewidth}
        \includegraphics[width=\linewidth]{../plots/tas_ACCESS-ESM1.5_esm-ssp585_ensembles.png}
    \end{subfigure}
    \begin{subfigure}[b]{0.45\linewidth}
        \includegraphics[width=\linewidth]{../plots/tas_ACCESS-ESM1.5_esm-ssp585_ensembles_diff.png}
    \end{subfigure}
    \begin{subfigure}[b]{0.45\linewidth}
        \includegraphics[width=\linewidth]{../plots/pr_ACCESS-ESM1.5_esm-ssp585_ensembles.png}
    \end{subfigure}
    \begin{subfigure}[b]{0.45\linewidth}
        \includegraphics[width=\linewidth]{../plots/pr_ACCESS-ESM1.5_esm-ssp585_ensembles_diff.png}
    \end{subfigure}
    \caption{Climate in \textit{esm-ssp585-ssp126Lu}. Left: Absolute values from the forestation simulation (\textit{esm-ssp585-ssp126Lu}). Right: difference between the \textit{esm-ssp585-ssp126Lu} simulation and \textit{esm-ssp585}}
    \label{fig:climate}
\end{figure}

\subsection{Regional analysis}

\begin{figure}[H]
    \centering
    \begin{subfigure}[b]{0.4\linewidth}
        \includegraphics[width=\linewidth]{../plots/tas_amazonia_diff.png}
    \end{subfigure}
    \begin{subfigure}[b]{0.4\linewidth}
        \includegraphics[width=\linewidth]{../plots/tas_easternnorthamerica_diff.png}
    \end{subfigure}
    \begin{subfigure}[b]{0.4\linewidth}
        \includegraphics[width=\linewidth]{../plots/tas_eastasia_diff.png}
    \end{subfigure}
    \begin{subfigure}[b]{0.4\linewidth}
        \includegraphics[width=\linewidth]{../plots/tas_borealnorthamerica_diff.png}
    \end{subfigure}
    \begin{subfigure}[b]{0.4\linewidth}
        \includegraphics[width=\linewidth]{../plots/tas_centralafrica_diff.png}
    \end{subfigure}
    \caption{Mean 2m surface air temperature for each region in ACCESS-ESM1-5.}
    \label{fig:ACCESS_tas_regions}
\end{figure}

\end{document}
