% Note: Figures are plotted using .png files for now since SVG rendering is very
% slow for LaTeX. At a later date I will change the figures to .svg files. For example:
%\includesvg[width=\linewidth]{../plots/cVeg_aus_anom.svg}
% Preamble.
\documentclass[]{article}
\usepackage[a4paper, total={170mm,257mm}, left=20mm, top=20mm,]{geometry}
\usepackage{svg} % Use SVGs only for final draft.
\usepackage{float} % Allows the [H] option meaning "place figure here and only here"
%\usepackage[section]{placeins} % Defines /FloatBarrier so that figures can't move.
\usepackage{subcaption} % Allows subfigures and captions
\usepackage{lineno} % Enable line numbers.
\usepackage{textcomp}
\usepackage[style=numeric]{biblatex}
\addbibresource{afforestation_references.bib}
\linenumbers
\title{Limited mitigation potentials under a high emissions scenario: multi-model and single model ensembles}
\author{Tammas Loughran et al.}

\begin{document}

\maketitle

\begin{center}
    \Large
    \vspace{0.9cm}
    \textbf{Abstract}
\end{center}

\begin{itemize}
    \item Impacts of global scale forrestation are still poorly understood.
    \item Likely no impact of forestation on global climate under a high emissions and high global warming scenario. Climate is dominated by the warming signal.
    \item Evidence of some increase in local scale warm extremes.
    \item Evidence in a shift in the global carbon balance; increased uptake of carbon on the land of ~25 PgC by 2100 and a decrease in the uptake of carbon by the ocean.
\end{itemize}

\section{Introduction}

A considerable amount of anthropogenic carbon emissions comes from agricultural activities.
In order to effectively limit global warming to below 2°C (or even 3°C), there must be a shift in the reliance of global diets away from agricultural practices that emit greenhouse gasses into the atmosphere (via e.g. forest clearing or ruminating herbivores).
This shift typically implies widespread abandonment of agricultural lands and an increase in natural vegetation (known as reforestation and afforestation of lands were not previously occupied by forest).
This study is motivated by existing studies on Forestation by Sonntag et al. (2016) \cite{sonntag_reforestation_2016} where the potential of forestation is quantified using and Earth System Model, where a high CO$_2$ emissions scenario is simulated (taken from RCP8.5) in combination with the "forestation" land-use from a low emissions scenario (taken from RCP4.5).
This previous study used the concentration driven CMIP5 version of the MPI-ESM and the CMIP5 representative concentration pathways.
This study aims to:
\begin{itemize}
    \item Quantify the CO$_2$ removal potential of the forestation strategy using CMIP6 models, both at the global and regional scale.
\end{itemize}
Furthermore, the climate variability uncertainties are quantified from a single model ensemble of the ACCES-ESM1.5 model.

\begin{itemize}
    \item As part of the contributions to the CMIP6 LUMIP experiments, modelling groups were requested to do emissions driven simulations of a similar forestation scenario. This time using the Shared Socioeconomic Pathways' emissions from SSP5-8.5 and the land use of SSP1-2.6.
    \item existing studies on MPI-ESM CMIP5 version using RCP8.5 emissions with RCP4.5 land use. \cite{sonntag_reforestation_2016} \cite{sonntag_quantifying_2018}
    \item ACCESS-ESM1.5 contributed 10 ensemble members to these experiments, and it was the only model to contribute multiple ensemble members to the CMIP6 afforestation experiment.
\end{itemize}

\section{Methods}

\subsection{Experiments}

The CMIP6 experimental design specifies a set of standard simulations called the Diagnostic, Evaluation and Characterization of Klima (DECK) \cite{eyring_overview_2016}, which are typically used as the foundation to study more specific research questions.
The DECK includes a pre-industrial control simulation (\textit{piControl}) with constant greenhouse gas concentration forcings, and a historical (\textit{historical}) simulation with transient historical greenhouse gas concentration forcings.
Furtermore, the DECK specifies corresponding simulations (\textit{esm-piControl} and \textit{esm-hist}) that are run in full "Earth system model" mode with a fully interactive dynamical carbon cycle and for the historical period, fossil fuel and industrial greenhouse gas emissions forcings.
In order to capture the carbon cycle and dynamics of the entire Earth system in response to forestation, including biogeochemical feedbacks, we use only the Earth system model simulations from relevant intercomparison projects that are based on these emission-driven simulations.

The forrestation scenario we use is the Land Use Model Intercomparison Project's \textit{esm-ssp585-ssp126Lu}, spanning the years 2015--2100 \cite{lawrence_land_2016}.
This simulation features the surface fossil fuel related CO$_{2}$ emissions from the SSP5-8.5 scenario, and the land-use change from the SSP1-2.6 scenario \cite{oneill_scenario_2016}.
The SSP5-8.5 scenario emissions represent a high emissions future pathway that results in a radiative forcing of 8.5 W m$^{-2}$ in 2100.
The SSP1-2.6 scenario land-use change assumes a future of sustainable development \cite{van_vuuren_energy_2017}.
This means low population growth, low pressures on land-use, environmentally friendly changes in diets and increased aggricultural efficiency and yields, resulting in abandonment of aggricultural lands.

The reference simulation we use to compare to \textit{esm-ssp585-ssp126Lu} is the \textit{esm-ssp585} from the Coupled Climate-Carbon Cycle Model Intercomparison Project (C4MIP) \cite{jones_c4mip_2016}.
This is the emissions-driven Earth system model simualtion corresponding to the SSP5-8.5 high greenhouse gas scenario.
In general, the land use change in this scenario has less expansion of forested lands than \textit{esm-ssp585-ssp126Lu}, but is a mixture of weak forestation and deforestation.
The notable difference in land-use change compared the forestation scenario is the lack of deforestation in the central African region (Figure S\ref{fig:ACCESS_land_cover}).
By taking the difference between the forestation simulation and the \textit{esm-ssp585} simulation (Equation \ref{equ:diff}), we can examine the impact of forestation on the climate and the carbon cycle.

\begin{itemize}
    \item esm-hist: emissions-driven simulations are initialized at 2015 using the esm-hist. \cite{eyring_overview_2016}
    \item esm-ssp585: High emissions scenario that assumes that development is driven by fossil fuels. \cite{oneill_scenario_2016}
    \item SSP1-2.6: Challenges to adaptation and mitigation are both low. Land use changes characterized by decrease in pasture and crop and increase in natural lands. \cite{van_vuuren_energy_2017}
\end{itemize}


\begin{equation}
    X|_{for} = X|_{esm-ssp585-ssp126Lu} - X|_{esm-ssp585}
    \label{equ:diff}
\end{equation}

\begin{figure}[H]
    \centering
    \begin{subfigure}[b]{0.45\linewidth}
        \includegraphics[width=\linewidth]{../plots/land_areas.png}
    \end{subfigure}
    \begin{subfigure}[b]{0.45\linewidth}

        \includegraphics[width=\linewidth]{../plots/land_areas_diff.png}
    \end{subfigure}
    \caption{Land use areas in ACCESS-ESM1.5. Left: ssp585 (dashed) and forestation scenario (solid). Right: difference between ssp585 and forestation scenario.}
    \label{fig:land_use}
\end{figure}

By the end of the century:
\begin{itemize}
    \item Forrest expands by: 1.59 million km²
    \item Abandonment: -1.11 million km²
\end{itemize}
Mid-century:
\begin{itemize}
    \item Crop minimum: -2.74 million km²
\end{itemize}

\begin{figure}[H]
    \centering
    \begin{subfigure}[b]{0.45\linewidth}
        \includegraphics[width=\linewidth]{../plots/treeFrac_anomaly.png}
    \end{subfigure}
    \begin{subfigure}[b]{0.45\linewidth}
        \includegraphics[width=\linewidth]{../plots/grassFrac_anomaly.png}
    \end{subfigure}
    \caption{ACCESS-ESM1.5 land-use area change from 2015 to 2100 in the forestation scenario. Left: forest. Right: Grass. In ACCESS-ESM1.5 LUH2 pasture fractions are allocated to C3 grass.}
    \label{fig:land_use_map}
\end{figure}

\begin{figure}[H]
    \centering
    \begin{subfigure}[b]{0.45\linewidth}
        \includegraphics[width=\linewidth]{../plots/CABLE_forests.png}
    \end{subfigure}
    \begin{subfigure}[b]{0.45\linewidth}
        \includegraphics[width=\linewidth]{../plots/CABLE_forests_deciduous_needle.png}
    \end{subfigure}
    \caption{CABLE tree PFTs for the SSP1-2.6 scenario.}
    \label{fig:CABLE_PFTs}
\end{figure}

Forestation is dominated by growth of evergreen broad leaf forests, followed by evergreen needle leaf forests, then deciduous broad leaf forests.
Deciduous needle leaf forests only make a small fraction of forests and do not show any expansion.
There are only a few areas where there is deforestation in the esm-ssp585-ssp126Lu experiment.
These occur in deciduous broad leaf forests in eastern USA, China and western Russia (Figure \ref{fig:land_use_map}).

\subsection{Participating models}

There are seven Earth system models that participated in both LUMIP and C4MIP with simulations available for \textit{esm-ssp585-ssp126Lu} and \textit{esm-ssp585}.
A brief overview of these models are presented in Table \ref{tab:models}.
2 models included wildfire schemes, 3 models included dynamical vegetation, and 6 models included nitrogen nutrient limitaiton.
ACCESS-ESM1-5 is the only model to include phosphorus limitation.
Furthermore, ACCESS-ESM1-5 is the only model to have multiple ensemble members available for both simulations, of which there are 10 each.

\begin{table}[H]
    \begin{tabular}{llllllll}
        \hline
Model         & Fire & Nitrogen & Phosphorus & Dynamic Veg. & No. PFTs & No. natural PFTs & Reference                                         \\ \hline
ACCESS-ESM1-5 & No   & Yes      & Yes        & No                 & 10       & 9                   & \cite{ziehn_australian_2020}   \\
BCC-CSM2-MR   & No   & Yes      & No         & No                 & 14       & 11                  & \cite{li_development_2019}     \\
CanESM5       & No   & No       & No         & Yes                & 9        & 7                   & \cite{swart_canadian_2019}     \\
MIROC-ES2L    & No   & Yes      & No         & No                 & 14       & 13                  & \cite{hajima_development_2020} \\
MPI-ESM1-2-LR & Yes  & Yes      & No         & Yes                & 13       & 12                  & \cite{giorgetta_climate_2013}  \\
NorESM2-LM    & Yes  & Yes      & No         & No                 & 22       & 14                  & \cite{seland_norwegian_2020}   \\
UKESM1-0-LL   & No   & Yes      & No         & Yes                & 12       & 8                   & \cite{sellar_ukesm1_2019}      \\ \hline
    \end{tabular}
    \caption{Models participating in both LUMIP and C4MIP with simulaions for both \textit{esm-ssp585-ssp126Lu} and \textit{esm-ssp585}. The columns indicate whether the models land surface component have representations of wildfire, nutrients, dynamical vegetation and the number of plant functional types. Natural plant functional types excludes aggricultural crops and pasture.}
    \label{tab:models}
\end{table}

\subsection{Regional analysis}

\begin{figure}[H]
    \centering
    \includegraphics{../plots/regional_analysis_map.png}
    \caption{Box regions for regional analysis.}
    \label{fig:box_regions}
\end{figure}

\section{Results}

\begin{itemize}
    \item Carbon
    \item Climate
    \item Regional analysis
\end{itemize}

\subsection{Model Inter-comparison}

\begin{figure}[H]
    \centering
    \begin{subfigure}[b]{0.45\linewidth}
        \includegraphics[width=\linewidth]{../plots/cLand_model_intercomparison_diff.png}
    \end{subfigure}
    \begin{subfigure}[b]{0.45\linewidth}
        \includegraphics[width=\linewidth]{../plots/cVeg_model_intercomparison_diff.png}
    \end{subfigure}
    \begin{subfigure}[b]{0.45\linewidth}
        \includegraphics[width=\linewidth]{../plots/cLitter_model_intercomparison_diff.png}
    \end{subfigure}
    \begin{subfigure}[b]{0.45\linewidth}
        \includegraphics[width=\linewidth]{../plots/cSoil_model_intercomparison_diff.png}
    \end{subfigure}
    \caption{cLand, cVeg, cLitter and cSoil in esm-ssp585-ssp126Lu scenario for 6 CMIP6 models.}
    \label{fig:models_cpools}
\end{figure}

\begin{itemize}
    \item CanESM5 has differing ensemble methods for esm-ssp585 and esm-ssp585-ssp126Lu. One uses r1i1p1f1 and another uses r1i1p2f1. Therefore, there is some additional perturbation applied to one of those experiments which causes the blue line to start at negative values.
    \item ACCESS-ESM is roughly in the missgle of the spread. More similar to MPI-ESM and UKESM than other models.
\end{itemize}

\subsection{ACCESS-ESM1.5}

\subsubsection{Climate}

\begin{itemize}
    \item Climate warms about 4 \textcelsius{} by 2100 and precipitation increases by 0.25e-5 kg m-2 s-1 \footnote{Need to change the units of this to something more human readable}
    \item There isn't much difference between the forestation scenario and ssp585.
\end{itemize}

\begin{figure}[H]
    \centering
    \begin{subfigure}[b]{0.45\linewidth}
        \includegraphics[width=\linewidth]{../plots/tas_ACCESS-ESM1-5_esm-ssp585_ensembles.png}
    \end{subfigure}
    \begin{subfigure}[b]{0.45\linewidth}
        \includegraphics[width=\linewidth]{../plots/tas_ACCESS-ESM1-5_esm-ssp585_ensembles_diff.png}
    \end{subfigure}
    \begin{subfigure}[b]{0.45\linewidth}
        \includegraphics[width=\linewidth]{../plots/pr_ACCESS-ESM1-5_esm-ssp585_ensembles.png}
    \end{subfigure}
    \begin{subfigure}[b]{0.45\linewidth}
        \includegraphics[width=\linewidth]{../plots/pr_ACCESS-ESM1-5_esm-ssp585_ensembles_diff.png}
    \end{subfigure}
    \caption{Climate in esm-ssp585-ssp126Lu. Left: Absolute values from the forestation simulation (esm-ssp585-ssp126Lu). Right: difference between esm-ssp585-ssp126Lu simulation and esm-ssp585}
    \label{fig:climate}
\end{figure}

\subsubsection{Carbon cycle}

\begin{itemize}
    \item There is an increase in vegetation carbon of about 25 Pg C by the end of the century.
    \item The increased CO$_2$ draw-down from the combined effect of forestation and CO$_2$ fertilization seems to be offset by the increase in respiration in the litter and soil by the end of the century.
    \item cSoil is lower than SSP585 by about 4 Pg carbon.
\end{itemize}

\begin{table}[H]
    \centering
    \begin{tabular}{@{}llll@{}}
    \hline
        & $\Delta$ Carbon (Pg) & \% present day & \% of 2080-2100 ssp585 \\ \hline
cLand   & 22.0±5.3    & 1.3±0.3        & 1.3±0.3                \\
cVeg    & 24.8±2.7    & 3.7±0.4        & 3.3±0.4                \\
cLitter & 0.4±0.3     & 0.8±0.6        & 0.9±0.7                \\
cSoil   & -4.3±2.3    & -0.5±0.3       & -0.5±0.3               \\ \hline
    \end{tabular}
    \caption{Cpools expressed as percentage of the present day carbon content or the end of ssp585}
    \label{tab:cpools_table}
\end{table}

\begin{figure}[H]
    \centering
    \begin{subfigure}[b]{0.4\linewidth}
        \includegraphics[width=\linewidth]{../plots/cVeg_ACCESS-ESM1-5_esm-ssp585-ssp126Lu_ensembles_anomalies.png}
    \end{subfigure}
    \begin{subfigure}[b]{0.4\linewidth}
        \includegraphics[width=\linewidth]{../plots/cVeg_ACCESS-ESM1-5_esm-ssp585-ssp126Lu_ensembles_diff.png}
    \end{subfigure}
    \begin{subfigure}[b]{0.4\linewidth}
        \includegraphics[width=\linewidth]{../plots/cLitter_ACCESS-ESM1-5_esm-ssp585-ssp126Lu_ensembles_anomalies.png}
    \end{subfigure}
    \begin{subfigure}[b]{0.4\linewidth}
        \includegraphics[width=\linewidth]{../plots/cLitter_ACCESS-ESM1-5_esm-ssp585-ssp126Lu_ensembles_diff.png}
    \end{subfigure}
    \begin{subfigure}[b]{0.4\linewidth}
        \includegraphics[width=\linewidth]{../plots/cSoil_ACCESS-ESM1-5_esm-ssp585-ssp126Lu_ensembles_anomalies.png}
    \end{subfigure}
    \begin{subfigure}[b]{0.4\linewidth}
        \includegraphics[width=\linewidth]{../plots/cSoil_ACCESS-ESM1-5_esm-ssp585-ssp126Lu_ensembles_diff.png}
    \end{subfigure}
\begin{subfigure}[b]{0.4\linewidth}
        \includegraphics[width=\linewidth]{../plots/cLand_ACCESS-ESM1.5_esm-ssp585-ssp126Lu_ensembles_anomalies.png}
    \end{subfigure}
    \begin{subfigure}[b]{0.4\linewidth}
        \includegraphics[width=\linewidth]{../plots/cLand_ACCESS-ESM1.5_esm-ssp585-ssp126Lu_ensembles_diff.png}
    \end{subfigure}
    \caption{Carbon pools. Left: esm-ssp585-ssp126Lu relative to the 2005–2025 period. Right: difference between the esm-ssp585-ssp126Lu and esm-ssp585.}
    \label{fig:cpools}
\end{figure}

\begin{figure}[H]
    \centering
    \begin{subfigure}[b]{0.4\linewidth}
        \includegraphics[width=\linewidth]{../plots/gpp_ACCESS-ESM1-5_esm-ssp585-ssp126Lu_ensembles_anomalies.png}
    \end{subfigure}
    \begin{subfigure}[b]{0.4\linewidth}
        \includegraphics[width=\linewidth]{../plots/gpp_ACCESS-ESM1-5_esm-ssp585-ssp126Lu_ensembles_diff.png}
    \end{subfigure}
    \begin{subfigure}[b]{0.4\linewidth}
        \includegraphics[width=\linewidth]{../plots/npp_ACCESS-ESM1-5_esm-ssp585-ssp126Lu_ensembles_anomalies.png}
    \end{subfigure}
    \begin{subfigure}[b]{0.4\linewidth}
        \includegraphics[width=\linewidth]{../plots/npp_ACCESS-ESM1-5_esm-ssp585-ssp126Lu_ensembles_diff.png}
    \end{subfigure}
    \begin{subfigure}[b]{0.4\linewidth}
        \includegraphics[width=\linewidth]{../plots/ra_ACCESS-ESM1-5_esm-ssp585-ssp126Lu_ensembles_anomalies.png}
    \end{subfigure}
    \begin{subfigure}[b]{0.4\linewidth}
        \includegraphics[width=\linewidth]{../plots/ra_ACCESS-ESM1-5_esm-ssp585-ssp126Lu_ensembles_diff.png}
    \end{subfigure}
    \begin{subfigure}[b]{0.4\linewidth}
        \includegraphics[width=\linewidth]{../plots/rh_ACCESS-ESM1-5_esm-ssp585-ssp126Lu_ensembles_anomalies.png}
    \end{subfigure}
    \begin{subfigure}[b]{0.4\linewidth}
        \includegraphics[width=\linewidth]{../plots/rh_ACCESS-ESM1-5_esm-ssp585-ssp126Lu_ensembles_diff.png}
    \end{subfigure}
    \begin{subfigure}[b]{0.4\linewidth}
        \includegraphics[width=\linewidth]{../plots/nbp_ACCESS-ESM1-5_esm-ssp585-ssp126Lu_ensembles_anomalies.png}
    \end{subfigure}
    \begin{subfigure}[b]{0.4\linewidth}
        \includegraphics[width=\linewidth]{../plots/nbp_ACCESS-ESM1-5_esm-ssp585-ssp126Lu_ensembles_diff.png}
    \end{subfigure}
    \caption{Carbon fluxes in esm-ssp585-ssp126Lu.  Left: relative to 2005–2025. Right: relative to ssp-585.}
    \label{fig:cflux}
\end{figure}

\begin{figure}[H]
    \centering
    \begin{subfigure}[b]{0.4\linewidth}
        \includegraphics[width=\linewidth]{../plots/CO2_aff_and_ssp585.png}
    \end{subfigure}
    \begin{subfigure}[b]{0.4\linewidth}
        \includegraphics[width=\linewidth]{../plots/carbon_budget_esm-ssp585-ssp126Lu.png}
    \end{subfigure}
    \caption{Top left: Atmospheric CO$_2$ concentration in the forestation scenario and ssp585. Top right: carbon flux budget in the forestation scenario.}
    \label{fig:atmosphere_carbon}
\end{figure}

\begin{figure}[H]
    \centering
    \begin{subfigure}[b]{0.4\linewidth}
        \includegraphics[width=\linewidth]{../plots/fgco2_ocean_carbon_aff_ssp585_flux.png}
    \end{subfigure}
    \begin{subfigure}[b]{0.4\linewidth}
        \includegraphics[width=\linewidth]{../plots/fgco2_ocean_carbon_aff_ssp585_cumulative.png}
    \end{subfigure}
    \caption{Left: Difference between the forestation scenario and ssp585 in downward ocean carbon mass flux. Right: Difference between the forestation scenario and ssp585 for the cumulative downward ocean carbon mass flux}
    \label{fig:ocean_carbon}
\end{figure}


\subsubsection{Regional analysis}

\begin{itemize}
    \item Australian cVeg increases by ~0.25 PgC in the forestation scenario. But the difference from the reference simulation is 0.
\end{itemize}

\begin{figure}[H]
    \centering
    \begin{subfigure}[b]{0.4\linewidth}
        \includegraphics[width=\linewidth]{../plots/cVeg_australia_anom.png}
    \end{subfigure}
    \begin{subfigure}[b]{0.4\linewidth}
        \includegraphics[width=\linewidth]{../plots/cVeg_australia_diff.png}
    \end{subfigure}
    \begin{subfigure}[b]{0.4\linewidth}
        \includegraphics[width=\linewidth]{../plots/cLitter_australia_anom.png}
    \end{subfigure}
    \begin{subfigure}[b]{0.4\linewidth}
        \includegraphics[width=\linewidth]{../plots/cLitter_australia_diff.png}
    \end{subfigure}
    \begin{subfigure}[b]{0.4\linewidth}
        \includegraphics[width=\linewidth]{../plots/cSoil_australia_anom.png}
    \end{subfigure}
    \begin{subfigure}[b]{0.4\linewidth}
        \includegraphics[width=\linewidth]{../plots/cSoil_australia_diff.png}
    \end{subfigure}
    \caption{Left: Australian region absolute carbon stocks. Right: Australian region  difference between forestation scenario and ssp585.}
    \label{fig:aus_region}
\end{figure}

\begin{itemize}
    \item The esm-ssp585-ssp126Lu land use has bot forestation and deforestation.
    \item Forestation occurs in Boreal Eurasia, Boreal North America and Brazil.
    \item Forestation and deforestation occurs in eastern North America, Central Africa and East Asia.
    \item Deforestation occurs in western Eurasia.
    \item Low forestation/deforestation occurs in Australia.
    \item No appreciable difference in climate variables.
    \item Increases in cVeg for all regions except Boreal North America, where some ensemble members show decreases in cVeg by the end of the century.
    \item Decreases in cSoil for Australia, Amazonia, Eastern North America. Western Eurasia, East Asia, Boreal Eurasia, Boreal North America, generally also show decreases in cSoil but there are a few ensemble members that show increases by the end of the century.
\end{itemize}

\begin{figure}[H]
    \centering
    \begin{subfigure}[b]{0.4\linewidth}
        \includegraphics[width=\linewidth]{../plots/cVeg_amazonia_diff.png}
    \end{subfigure}

    \begin{subfigure}[b]{0.4\linewidth}
        \includegraphics[width=\linewidth]{../plots/cVeg_easternnorthamerica_diff.png}
    \end{subfigure}
    \begin{subfigure}[b]{0.4\linewidth}
        \includegraphics[width=\linewidth]{../plots/cVeg_eastasia_diff.png}
    \end{subfigure}
    \begin{subfigure}[b]{0.4\linewidth}
        \includegraphics[width=\linewidth]{../plots/cVeg_borealnorthamerica_diff.png}
    \end{subfigure}
    \begin{subfigure}[b]{0.4\linewidth}
        \includegraphics[width=\linewidth]{../plots/cVeg_centralafrica_diff.png}
    \end{subfigure}
    \caption{Experiment differences for each region in \ref{fig:box_regions}}
    \label{fig:aus_region_cveg_tas}
\end{figure}

\subsection{Notes/Discussion}

\begin{itemize}
    \item The lack of climate warming mitigation potential in the forestation scenario may be due to redistribution of the global carbon budget into the natural sinks.
        An increase in vegetation uptake would lead to a decrease in atmospheric CO$_2$ accumulation, which would result in a reduction of partial CO$_2$ pressure on the ocean and hence lead to a reduction of CO$_2$ in the ocean sink.
    \item The LUH2 data set severely underestimates the tree cover that is originally dictated by IMAGE (the integrated assessment model that produces the SSP1-2.6 scenario).
    \item The bio-geophysical impacts of growing trees causes localized warming at the extreme ends of the temperature distribution, especial in the higher latitudes where trees tend to be slow growing and the bare ground either has a high albedo or is covered in snow. This warming might offset the climate mitigation potential of forestation.
\end{itemize}

\section{Conclusion?}

\printbibliography

\section{Supplementary Material}


\subsection{PFTs}

\begin{figure}[H]
    \centering
    \begin{subfigure}[b]{0.4\linewidth}
        \includegraphics[width=\linewidth]{../plots/treeFrac_esm-ssp585_anomaly.png}
    \end{subfigure}
    \begin{subfigure}[b]{0.4\linewidth}
        \includegraphics[width=\linewidth]{../plots/grassFrac_esm-ssp585_anomaly.png}
    \end{subfigure}
    \begin{subfigure}[b]{0.4\linewidth}
        \includegraphics[width=\linewidth]{../plots/cropFrac_esm-ssp585_anomaly.png}
    \end{subfigure}
    \caption{Difference in area of trees, grass and crop between 2015 and 2100 for esm-ssp585 in ACCESS-ESM1.5}
    \label{fig:ACCESS_land_cover}
\end{figure}

\subsection{Regional analysis}

\begin{figure}[H]
    \centering
    \begin{subfigure}[b]{0.4\linewidth}
        \includegraphics[width=\linewidth]{../plots/tas_amazonia_diff.png}
    \end{subfigure}
    \begin{subfigure}[b]{0.4\linewidth}
        \includegraphics[width=\linewidth]{../plots/tas_easternnorthamerica_diff.png}
    \end{subfigure}
    \begin{subfigure}[b]{0.4\linewidth}
        \includegraphics[width=\linewidth]{../plots/tas_eastasia_diff.png}
    \end{subfigure}
    \begin{subfigure}[b]{0.4\linewidth}
        \includegraphics[width=\linewidth]{../plots/tas_borealnorthamerica_diff.png}
    \end{subfigure}
    \begin{subfigure}[b]{0.4\linewidth}
        \includegraphics[width=\linewidth]{../plots/tas_centralafrica_diff.png}
    \end{subfigure}
    \caption{Mean surface air temperature for each region in ACCESS-ESM1-5.}
    \label{fig:ACCESS_tas_regions}
\end{figure}

\end{document}

